%%% generic article type (pdf)latex file
%%% use together with Makefile

\documentclass[letterpaper]{scrartcl}
\usepackage{graphicx}
\usepackage{amsmath,amsfonts,amsthm}
\usepackage{eufrak}
\usepackage{mathabx}
\usepackage{url}
\usepackage[usenames,dvipsnames,svgnames,table]{xcolor}
\usepackage[colorlinks]{hyperref}
\hypersetup{
     colorlinks   = true,
     urlcolor     = blue,
     linkcolor    = red,
     citecolor    = black
}
\usepackage{enumitem}
\usepackage{booktabs}
\usepackage{cprotect}
\usepackage{minted}


%\usepackage{wrapfig}
%\usepackage{subfig}
%\usepackage[format=plain,labelsep=period,font=small,labelfont=bf]{caption}

%------------------------------------------------------------
% assignment
%
\newcommand{\anumber}{3}
%
%------------------------------------------------------------

% hyperref https://en.wikibooks.org/wiki/LaTeX/Hyperlinks#.5Chref
\urlstyle{same}

%% not working yet...
\newcounter{TotalPoints}
\newcounter{TotalBonus}

\newcommand{\BONUS}{\textsc{Bonus: }}
\newcommand{\bonus}[1]{\textbf{[bonus +#1*]}\stepcounter{TotalBonus}}
\newcommand{\points}[1]{\textbf{[#1 points]}\stepcounter{TotalPoints}}
\newenvironment{enuma}{\begin{enumerate}[label=(\alph*)]}{\end{enumerate}}
\newenvironment{enumi}{\begin{enumerate}[label=(\roman*)]}{\end{enumerate}}
\newenvironment{solution}{\par\noindent\P{} }{\ \qedsymbol}

\renewcommand{\vec}[1]{\ensuremath{\mathbf{#1}}}
\newcommand{\pd}[3][]{\left(\frac{\partial #2}{\partial #3}\right)_{#1}}

\newcommand{\python}[1]{\mintinline{python}{#1}}

\newcommand{\anum}{0\anumber}


\begin{document}
%\maketitle

\setcounter{section}{\anumber}
\addtocounter{section}{-1}
\section{ --- PHY 494: Homework assignment (52 points total)}

\noindent Due Thursday, Jan 31, 2019, 5pm.

\noindent
\fbox{\parbox{\linewidth}{Submission is now to your \textbf{private
      GitHub repository}. Follow the link provided to you by the
    instructor in order for the repository to be set up: It will have
    the name
    \emph{ASU-CompMethodsPhysics-PHY494/assignments-2019-\emph{YourGitHubUsername}}
    and will only be visible to you and the instructor/TA. Follow the
    instructions below to submit this (and all future) homework.}}
Read the following instructions carefully. Ask if anything is unclear.
\begin{enumerate}
\item \texttt{git clone} your assignment repository (change
  \emph{YourGitHubUsername} to your GitHub username)
  \begin{minted}{bash}
    repo="assignments-2019-YourGitHubUsername.git" 
    git clone https://github.com/ASU-CompMethodsPhysics-PHY494/${repo}
  \end{minted}
\item run the script
  \texttt{./scripts/update.sh} (replace \emph{YourGitHubUsername} with
  your GitHub username):
  \begin{minted}{bash}
    cd ${repo} 
    bash ./scripts/update.sh
  \end{minted}
  It should create three subdirectories\footnote{If the script fails,
    file an issue in the
    \href{https://github.com/ASU-CompMethodsPhysics-PHY494/PHY494-assignments-skeleton/issues}{Issue
      Tracker for PHY494-assignments-skeleton} and just create the
    directories manually.} \texttt{assignment\_\anum/Submission},
  \texttt{assignment\_\anum/Grade}, and
  \texttt{assignment\_\anum/Work}.
\item You can try out code in the \texttt{assignment\_\anum/Work}
  directory but you don't have to use it if you don't want to. Your
  grade with comments will appear in
  \texttt{assignment\_\anum/Grade}.
\item Create your solution in
  \texttt{assignment\_\anum/Submission}. Use Git to \texttt{git
    add} files and \texttt{git commit} changes.

  You can create a PDF, a text file or Jupyter notebook inside the
  \texttt{assignment\_\anum/Submission} directory as well as Python
  code (if required). \textbf{Name your files \texttt{hw\anum.pdf} or
    \texttt{hw\anum.txt} or \texttt{hw\anum.ipynb}}, depending on how
  you format your work. Files with code (if requested) should be named
  exactly as required in the assignment.
\item When you are ready to submit your solution, do a final
  \texttt{git status} to check that you haven't forgotten anything,
  commit any uncommited changes, and \texttt{git push} to your GitHub
  repository. Check on \emph{your} GitHub repository web
  page\footnote{\texttt{https://github.com/ASU-CompMethodsPhysics-PHY494/assignments-2019-\emph{YourGitHubUsername}}}
  that your files were properly submitted.

  You can push more updates up until the deadline. Changes after the
  deadline will not be taken into account for grading.
\end{enumerate}
Homeworks must be legible and intelligible and on-time or may  be
returned ungraded with 0 points.

This assignment contains \textbf{bonus problems}. A bonus problem is
optional. If you do it you get additional points that count towards
this homework's total, although you can't get more than the maximum
number of points. If you don't do it you can still get full
points. Bonus problems and bonus points are indicated with an asterisk
``*''.


\subsection{Python data types (7 points)}
What is the Python data type of each of the following values?
\begin{enuma}
\item \python{3.14515} \points{1}
\item \python{0} \points{1}
\item \python{False} \points{1}
\item \python{'To be or not to be'} \points{1}
\item \python{[3, 2, 1, "lift off!"]} \points{1}
\item \python{3, 2, 1, "lift off!"} \points{1}
\item \python{None} \points{1}
\item \BONUS \python{{'name': 'Hamlet', 'occupation': 'Prince'}} \bonus{1}
\end{enuma}


\subsection{Python Lists and Strings (20 points)}

Lists and strings share some similarities but also have important
differences. Let's look at them. (Type code in the Python interpreter
e.g., \texttt{ipython}).

\begin{minted}{python}
bag = ["guide", "towel", "tea", 42]
ga = "Four score and seven years ago"
\end{minted}

\begin{enuma}
\item How do you have to slice \python{bag} in order to get
  \python{['towel', 'tea']}? \points{1}
\item What does \python{bag[::-1]} do?  
  
  How do you slice bag in order to get
  \python{['tea', 'towel']}?  \points{2}
\item Strings can also be sliced. How do you have to slice \python{ga}
  to get
  \begin{itemize}
  \item   "Four"
  \item   "seven"
  \end{itemize}
  \points{2}
\item You can access elements of a list in a variety of ways: 
  \begin{enumi}
  \item   Explain what
\begin{minted}{python}
bag[0] = 'book'
\end{minted}
    does? (Hint: print \python{bag}!) \points{1}
  \item  Create two new variables:
\begin{minted}{python}
mybag = bag
yourbag = bag[:]
\end{minted}
    and use them:
\begin{minted}{python}
mybag[3] = "mice" 
yourbag.append("money")
\end{minted}
    What is the content of \python{bag}, \python{mybag},
    \python{yourbag}? \points{2}
  \item  From your observations, explain how the assignment \python{x = a} differs
    from \python{y = a[:]}? \points{3}
  \end{enumi}
\item  Try
\begin{minted}{python}
  ga[:4] = "Three"
\end{minted}
  \begin{enumi}
  \item Describe what happens?\footnote{Note that strings are
      ``immutable'' objects in Python whereas lists are ``mutable''.}
    \points{1}
  \item How would you construct the string \python{"Three score and
      seven years ago"} from \python{ga} and the string
    \python{"Three"}? \points{1}
  \end{enumi}
\item What do the commands
\begin{minted}{python}
ga.split()
a, b, c = ga.split()[:3]
list([1,2,3])
list(ga)
\end{minted}
  do? You can show the output but you need to explain in your own
  words what is happening. \points{4}
\item Nested lists: Given the list
\begin{minted}{python}
bags = [['salt', 'pepper'], ['pen', 'eraser', 'ruler']]
\end{minted}
  how do you have to index \python{bags} to get

  \begin{enumi}
  \item \python{['salt', 'pepper']} \points{1}
  \item \python{'pepper'} \points{1}
  \item \python{'ruler'} \points{1}
  \end{enumi}
\end{enuma}

\subsection{Loops (14 points)}
\begin{enuma}
\item Use a \python{for} loop to print each element of the list
  \begin{minted}{python}
    sentence = ["We", "must", "walk", "before", "we", "can", "run"]
  \end{minted}
  Show your code and your output. \points{4}
\item \BONUS Find a way to only print every other line from
  \python{sentence}, i.e., the output should read ``We walk we
  run''. Show your code and your output. \bonus{2}
\item Use a loop to sum the integers from 1 to 1000 and print the
  final result (500500). \points{4}

  \noindent\emph{Note}: You can increment a variable \python{total}
  with a value \python{x} with the assignment
  \begin{minted}{python}
    total = total + x
  \end{minted}
  or equivalently but more compactly written
  \begin{minted}{python}
    total += x
  \end{minted}
  Both expressions do the same thing: add two values together and then
  assign the sum to the variable \python{total}, overwriting the old
  value of \python{total} in the process.
\item Write a program that counts down from 10 to 0 and prints each
  number (``10 9 8 ... 3 2 1'').
  \begin{enumi}    
  \item Use a \python{while} loop. \points{3}
  \item Use a \python{for} loop. \points{3}
  \end{enumi}
  (You can use any other Python functions that you might need, such as
  \python{range}.)

  Show your code and output.
\end{enuma}


\subsection{Simple coordinate manipulation in Python (11 points)}

We can represent the cartesian coordinates
$\vec{r}_{i} = (x_{i}, y_{i}, z_{i})$ for four particles as a list of
lists \texttt{positions}:
\begin{minted}{python}
  positions = \
      [[0.0, 0.0, 0.0], [1.34234, 1.34234, 0.0], \
       [1.34234, 0.0,  1.34234], [0.0, 1.34234, 1.34234]]
\end{minted}
For the following, do not import any additional modules: \emph{only
  use pure Python.} The repository contains skeleton code
\texttt{coordinates\_a.py}, \texttt{coordinates\_b.py},
\texttt{coordinates\_c.py}, \texttt{coordinates\_d.py} for the
sub-problems. Add your code to these files and \emph{submit them as
  part of your solution}---code must be included to get full
points.\footnote{You will also see a file
  \texttt{test\_coordinates.py}. It contains \emph{tests} that check
  your code. Your instructors will \emph{run these tests on your
    code}. You can run them yourself with the \texttt{pytest} command,

  \mintinline{bash}{pytest -v test_coordinates.py}

  If everything is correct, you should see something like
  \texttt{===== 9 passed in 0.10 seconds =====}. If tests fail then
  you can correct your code until you get the tests to pass.}
\begin{enuma}
  \item Access the coordinates of the second particle, assign it to
    a variable \texttt{particle2}, and print the coordinates. \points{1}
  \item Access the $y$-coordinate of the
    second particle, assign it to the variable \texttt{y2}, and print
    the value. \points{1}
  \item \label{li:translation} Write Python code to translate all particles by a vector
    $\vec{t} = (1.34234, -1.34234, -1.34234)$,
\begin{minted}{python}
t = [1.34234, -1.34234, -1.34234]
\end{minted}
    Assign the translated coordinates to the variable
    \texttt{new\_positions} and print them. \points{3}
  \item Make your solution of \ref{li:translation} a function
    \texttt{translate(coordinates, t)}, which translates all
    coordinates in the argument \texttt{coordinates} (a list of $N$
    lists of length 3) by the translation vector in \texttt{t}. The
    function should return the translated coordinates.

    Apply your function to (1) the input \texttt{positions} and
    \texttt{t} from above and (2) for \texttt{positions2 = [[1.5,
      -1.5, 3], [-1.5, -1.5, -3]]} and \texttt{t = [-1.5, 1.5,
      3]} and show the output. \points{6}
\end{enuma}




%Total Points: \arabic{TotalPoints}. Total Bonus: \arabic{TotalBonus}*


\end{document}

%%% Local Variables: 
%%% mode: latex
%%% TeX-master: t
%%% End: 
