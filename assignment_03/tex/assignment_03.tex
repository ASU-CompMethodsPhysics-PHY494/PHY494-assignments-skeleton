%%% generic article type (pdf)latex file
%%% use together with Makefile

\documentclass[letterpaper]{scrartcl}
\usepackage{graphicx}
\usepackage{amsmath,amsfonts,amsthm}
\usepackage{eufrak}
\usepackage{mathabx}
\usepackage{url}
\usepackage[usenames,dvipsnames,svgnames,table]{xcolor}
\usepackage[colorlinks]{hyperref}
\hypersetup{
     colorlinks   = true,
     urlcolor     = blue,
     linkcolor    = red,
     citecolor    = black
}
\usepackage{enumitem}
\usepackage{booktabs}
\usepackage{cprotect}
\usepackage{minted}

% https://tex.stackexchange.com/a/340588
\usepackage{tikz}
\usepackage[edges]{forest}

%\usepackage{wrapfig}
%\usepackage{subfig}
%\usepackage[format=plain,labelsep=period,font=small,labelfont=bf]{caption}

%------------------------------------------------------------
% assignment
%
\newcommand{\anumber}{3}
%
%------------------------------------------------------------

% hyperref https://en.wikibooks.org/wiki/LaTeX/Hyperlinks#.5Chref
\urlstyle{same}

%% not working yet...
\newcounter{TotalPoints}
\newcounter{TotalBonus}

\newcommand{\BONUS}{\textsc{Bonus: }}
\newcommand{\bonus}[1]{\textbf{[bonus +#1*]}\stepcounter{TotalBonus}}
\newcommand{\points}[1]{\textbf{[#1 points]}\stepcounter{TotalPoints}}
\newenvironment{enuma}{\begin{enumerate}[label=(\alph*)]}{\end{enumerate}}
\newenvironment{enumi}{\begin{enumerate}[label=(\roman*)]}{\end{enumerate}}
\newenvironment{solution}{\par\noindent\P{} }{\ \qedsymbol}

\renewcommand{\vec}[1]{\ensuremath{\mathbf{#1}}}
\newcommand{\pd}[3][]{\left(\frac{\partial #2}{\partial #3}\right)_{#1}}

\newcommand{\anum}{0\anumber}


% dir trees from Guilherme Zanotelli
% https://tex.stackexchange.com/a/340588
\definecolor{foldercolor}{RGB}{124,166,198}

\tikzset{pics/folder/.style={code={%
    \node[inner sep=0pt, minimum size=#1](-foldericon){};
    \node[folder style, inner sep=0pt, minimum width=0.3*#1, minimum height=0.6*#1, above right, xshift=0.05*#1] at (-foldericon.west){};
    \node[folder style, inner sep=0pt, minimum size=#1] at (-foldericon.center){};}
    },
    pics/folder/.default={20pt},
    folder style/.style={draw=foldercolor!80!black,top color=foldercolor!40,bottom color=foldercolor}
}

\forestset{is file/.style={edge path'/.expanded={%
        ([xshift=\forestregister{folder indent}]!u.parent anchor) |- (.child anchor)},
        inner sep=1pt},
    this folder size/.style={edge path'/.expanded={%
        ([xshift=\forestregister{folder indent}]!u.parent anchor) |- (.child anchor) pic[solid]{folder=#1}}, inner xsep=0.6*#1},
    folder tree indent/.style={before computing xy={l=#1}},
    folder icons/.style={folder, this folder size=#1, folder tree indent=3*#1},
    folder icons/.default={12pt},
}


\begin{document}
%\maketitle

\setcounter{section}{\anumber}
\addtocounter{section}{-1}
\section{ --- PHY 494: Homework assignment (50 points total)}

\noindent Due Friday, Feb 9, 2018, 5pm.

\noindent
\fbox{\parbox{\linewidth}{Submission is now to your \textbf{private
      GitHub repository}. Follow the link provided to you by the
    instructor in order for the repository to be set up: It will have
    the name
    \emph{ASU-CompMethodsPhysics-PHY494/assignments-2018-\emph{YourGitHubUsername}}
    and will only be visible to you and the instructor/TA. Follow the
    instructions below to submit this (and all future) homework.}}
Read the following instructions carefully. Ask if anything is unclear.
\begin{enumerate}
\item \texttt{git clone} your assignment repository (change
  \emph{YourGitHubUsername} to your GitHub username)
  \begin{minted}{bash}
    repo="assignments-2018-YourGitHubUsername.git" 
    git clone https://github.com/ASU-CompMethodsPhysics-PHY494/${repo}
  \end{minted}
\item run the script
  \texttt{./scripts/update.sh} (replace \emph{YourGitHubUsername} with
  your GitHub username):
  \begin{minted}{bash}
    cd ${repo} 
    bash ./scripts/update.sh
  \end{minted}
  It should create three subdirectories\footnote{If the script fails,
    file an issue in the
    \href{https://github.com/ASU-CompMethodsPhysics-PHY494/PHY494-assignments-skeleton/issues}{Issue
      Tracker for PHY494-assignments-skeleton} and just create the
    directories manually.} \texttt{assignment\_\anum/Submission},
  \texttt{assignment\_\anum/Grade}, and
  \texttt{assignment\_\anum/Work}.
\item You can try out code in the \texttt{assignment\_\anum/Work}
  directory but you don't have to use it if you don't want to. Your
  grade with comments will appear in
  \texttt{assignment\_\anum/Grade}.
\item Create your solution in
  \texttt{assignment\_\anum/Submission}. Use Git to \texttt{git
    add} files and \texttt{git commit} changes.

  You can create a PDF, a text file or Jupyter notebook inside the
  \texttt{assignment\_\anum/Submission} directory as well as Python
  code (if required). \textbf{Name your files \texttt{hw\anum.pdf} or
    \texttt{hw\anum.txt} or \texttt{hw\anum.ipynb}}, depending on how
  you format your work. Files with code (if requested) should be named
  exactly as required in the assignment.
\item When you are ready to submit your solution, do a final
  \texttt{git status} to check that you haven't forgotten anything,
  commit any uncommited changes, and \texttt{git push} to your GitHub
  repository. Check on \emph{your} GitHub repository web
  page\footnote{\texttt{https://github.com/ASU-CompMethodsPhysics-PHY494/assignments-2018-\emph{YourGitHubUsername}}}
  that your files were properly submitted.

  You can push more updates up until the deadline. Changes after the
  deadline will not be taken into account for grading.
\end{enumerate}
Homeworks must be legible and intelligible and on-time or may  be
returned ungraded with 0 points.

% This assignment contains \textbf{bonus problems}. A bonus problem is
% optional. If you do it you get additional points that count towards
% this homework's total, although you can't get more than the maximum
% number of points. If you don't do it you can still get full
% points. Bonus problems and bonus points are indicated with an asterisk
% ``*''.

\subsection{Version control with Git (9 points)}
Keep your answers short, one or two sentences should be sufficient for
questions \ref{li:gitinit}--\ref{li:gitpush}.
\begin{enuma}
\item \emph{Briefly} explain what a version control system such as Git
  does and how it can help you. (For your answer, it is sufficient to
  focus on three different aspects out of many --- choose the ones
  that \emph{you} find most important.) \points{3}
\item \label{li:gitinit}Explain the difference between \texttt{git init} and \texttt{git
  clone}. \points{2}
\item Explain the difference between \texttt{git add} and \texttt{git
    commit}. \points{2}
\item \label{li:gitpush} Explain the difference between \texttt{git commit} and
  \texttt{git push} \points{2}
\end{enuma}

\subsection{Be the Git (10 points)}
In this problem you should state various outcomes when a number of git
commands are run on the  directory structure in Figure~\ref{fig:tree}
(\emph{Documents} is the top-level directory, sub-directories are
indicated by blue folder icons, files are text):\footnote{You can
  pretend to be git or you can also create the directory structure
  yourself and run the commands.}

\begin{figure}[hbt]
  \centering
  \begin{forest}
    for tree={font=\sffamily, grow'=0, folder indent=.9em, folder
      icons, edge=densely dotted} [Documents [planets [hoth.jpg, is
    file] [tatooine.jpg, is file] [alderaan.jpg, is file] ] [vehicles
    [ships [destroyer.txt, is file] [interceptor.txt, is file]]
    [stations [deathstar.xxx, is file]]] [jedi\_locations.map, is
    file] [shopping.txt, is file] [TODO, is file] ]
  \end{forest}  
  \caption{Part of the directory tree of user dvader. \emph{Documents}
    is the top-level directory, sub-directories are indicated by blue
    folder icons, files are just shown as text with their file name,
    e.g., \emph{TODO} or \emph{hoth.jpg}.}
  \label{fig:tree}
\end{figure}

User dvader decides (quite sensibly) to use version control, namely
\textbf{git}, for his Documents. The \emph{Documents} directory was
not under version control before.  The first set of commands that he
performs is:
\begin{minted}{bash}
cd Documents
git init
git add shopping.txt
git add planets/alderaan.jpg
\end{minted}

To answer the following question, create a table where you list the
command that is being executed, the \emph{files in the git staging
  area} and the \emph{files in the git repository}. After the initial
commands, your table should look like this:

\begin{center}
  \begin{tabular}{lp{0.4\linewidth}p{0.3\linewidth}}
    \toprule
    command & staging area & repository\\
    \midrule
    git init & & \\
    git add shopping.txt & shopping.txt & \\
    git add planets/alderaan.jpg & shopping.txt, planets/alderaan.jpg & \\
    \bottomrule                           
  \end{tabular}
\end{center}

List \emph{all} files that are in either staging area or repository,
not just new ones. If there are no files (as in the repository during
the first three \texttt{git} commands), leave the space empty.

Add the table above to your submitted solution and continue it by
listing all the files in the staging area and in the repository, after
each of the following commands have been carried out:
\begin{minted}{bash}
git commit
git add vehicles/stations
git commit
git add vehicles TODO
git commit
git rm planets/alderaan.jpg
git add planets
git commit
\end{minted}



\subsection{Your GitHub account (10 points)}

As part of the last lesson you should have
\href{http://asu-compmethodsphysics-phy494.github.io/ASU-PHY494/page2/#set-up-your-own-github-repositories}{set
  up your own GitHub account} on \url{https://github.com} (if you have
not done it yet, do it now!). What is your \textbf{GitHub username}?
\begin{itemize}
\item Write down your GitHub username. \points{10}
\item Take the survey
  \href{https://goo.gl/forms/eA0BZ4xMMijsY8Gp2}{PHY
    494: Your GitHub account}\footnote{In case the link to the survey
    is not clickable: got to
    \url{https://goo.gl/forms/AZMtF6c60FBdCCkt1}. You must be logged
    in with your ASU account. Log in to \url{https://my.asu.edu} first
    and then go to the survey.} if you have not done so already.
\end{itemize}


\subsection{Simple coordinate manipulation in Python (11 points)}

We can represent the cartesian coordinates
$\vec{r}_{i} = (x_{i}, y_{i}, z_{i})$ for four particles as a list of
lists \texttt{positions}:
\begin{minted}{python}
  positions = \
      [[0.0, 0.0, 0.0], [1.34234, 1.34234, 0.0], \
       [1.34234, 0.0,  1.34234], [0.0, 1.34234, 1.34234]]
\end{minted}
For the following, do not import any additional modules: \emph{only
  use pure Python.} The repository contains skeleton code
\texttt{coordinates\_a.py}, \texttt{coordinates\_b.py},
\texttt{coordinates\_c.py}, \texttt{coordinates\_d.py} for the
sub-problems. Add your code to these files and \emph{submit them as
  part of your solution}---code must be included to get full
points.\footnote{You will also see a file
  \texttt{test\_coordinates.py}. It contains \emph{tests} that check
  your code. Your instructors will \emph{run these tests on your
    code}. You can run them yourself with the \texttt{pytest} command,

  \mintinline{bash}{pytest -v test_coordinates.py}

  If everything is correct, you should see something like
  \texttt{===== 9 passed in 0.10 seconds =====}. If tests fail then
  you can correct your code until you get the tests to pass.}
\begin{enuma}
  \item Access the coordinates of the second particle, assign it to
    a variable \texttt{particle2}, and print the coordinates. \points{1}
  \item Access the $y$-coordinate of the
    second particle, assign it to the variable \texttt{y2}, and print
    the value. \points{1}
  \item \label{li:translation} Write Python code to translate all particles by a vector
    $\vec{t} = (1.34234, -1.34234, -1.34234)$,
\begin{minted}{python}
t = [1.34234, -1.34234, -1.34234]
\end{minted}
    Assign the translated coordinates to the variable
    \texttt{new\_positions} and print them. \points{3}
  \item Make your solution of \ref{li:translation} a function
    \texttt{translate(coordinates, t)}, which translates all
    coordinates in the argument \texttt{coordinates} (a list of $N$
    lists of length 3) by the translation vector in \texttt{t}. The
    function should return the translated coordinates.

    Apply your function to (1) the input \texttt{positions} and
    \texttt{t} from above and (2) for \texttt{positions2 = [[1.5,
      -1.5, 3], [-1.5, -1.5, -3]]} and \texttt{t = [-1.5, 1.5,
      3]} and show the output. \points{6}
\end{enuma}

\subsection{Squash the bug (10 points)}
\label{sec:bugs}

Three files \texttt{bug\_a.py}, \texttt{bug\_b.py}, and
\texttt{bug\_c.py} are include. Each contains at least one Python
bug. Fix them and commit your fixed files.\footnote{You will also see a file
  \texttt{test\_bugs.py}. It contains \emph{tests} that check
  your code. Your instructors will \emph{run these tests on your
    code}. You can run them yourself with the \texttt{pytest} command,

  \mintinline{bash}{pytest -v test_bugs.py}

  If everything is correct, you should see something like
  \texttt{===== 36 passed in 0.36 seconds =====}. If tests fail then
  you can correct your code until you get the tests to pass.}
\begin{enuma}
\item Fix \texttt{bug\_a.py} and commit the fixed file. The code
  should run, assign the correct value to the variable \texttt{value},
  and print the correct value. \points{3}
\item Fix \texttt{bug\_b.py} and commit the fixed file. The code
  should add two vectors $\vec{a}= (12.3, 3.90, 4.5)$ and
  $\vec{b} = (1.3, 0.91, -3.3)$ and print the value of the new vector
  $\vec{c} = 5\vec{a} - 3\vec{b}$ and assign the value to the variable
  \texttt{c}. \points{3}
\item Fix \texttt{bug\_c.py} and commit the fixed file. You should
  correctly implement the 2D $\mathrm{sinc}$ function
  \begin{gather*}
    \mathrm{sinc}(x, y) := \frac{\sin x}{x} \frac{\sin y}{y}
  \end{gather*}
  (where $x$ and $y$ are given in radians). The function should be
  correct for arbitrary input.\footnote{Hint: Especially consider the edge
    and corner cases.} \points{4}
\end{enuma}



%Total Points: \arabic{TotalPoints}. Total Bonus: \arabic{TotalBonus}*


\end{document}

%%% Local Variables: 
%%% mode: latex
%%% TeX-master: t
%%% End: 
