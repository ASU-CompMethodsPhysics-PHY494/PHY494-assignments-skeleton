%%% generic article type (pdf)latex file
%%% use together with Makefile

\documentclass[paper=letter]{scrartcl}
\usepackage{graphicx}
\usepackage{amsmath,amsfonts,amsthm}
\usepackage{eufrak}
\usepackage{mathabx}
\usepackage{url}
\usepackage[usenames,dvipsnames,svgnames,table]{xcolor}
\usepackage[colorlinks]{hyperref}
\hypersetup{
     colorlinks   = true,
     urlcolor     = blue,
     linkcolor    = red,
     citecolor    = black
}
\usepackage{enumitem}
\usepackage{booktabs}
\usepackage{cprotect}
\usepackage{minted}

%\usepackage{wrapfig}
%\usepackage{subfig}
%\usepackage[format=plain,labelsep=period,font=small,labelfont=bf]{caption}

%------------------------------------------------------------
% assignment
%
\newcommand{\anumber}{7}
%
%------------------------------------------------------------

% hyperref https://en.wikibooks.org/wiki/LaTeX/Hyperlinks#.5Chref
\urlstyle{same}

%% not working yet...
\newcounter{TotalPoints}
\newcounter{TotalBonus}

\newcommand{\BONUS}{\textsc{Bonus: }}
\newcommand{\bonus}[1]{\textbf{[bonus +#1*]}\stepcounter{TotalBonus}}
\newcommand{\points}[1]{\textbf{[#1 points]}\stepcounter{TotalPoints}}
\newenvironment{enuma}{\begin{enumerate}[label=(\alph*)]}{\end{enumerate}}
\newenvironment{enumi}{\begin{enumerate}[label=(\roman*)]}{\end{enumerate}}
\newenvironment{solution}{\par\noindent\P{} }{\ \qedsymbol}

\renewcommand{\vec}[1]{\ensuremath{\mathbf{#1}}}
\newcommand{\pd}[3][]{\left(\frac{\partial #2}{\partial #3}\right)_{#1}}

\newcommand{\anum}{0\anumber}


\begin{document}
%\maketitle

\setcounter{section}{\anumber}
\addtocounter{section}{-1}
\section{ --- PHY 494: Homework assignment (28 points total)}

\noindent Due Thursday, March 5, 2020, 11:59pm.

\noindent
%  \url{}
\fbox{\parbox{\linewidth}{Submission is to your \textbf{private
      GitHub repository}.}}
Read the following instructions carefully. Ask if anything is unclear.
\begin{enumerate}
\item \texttt{cd} into your assignment repository (change
  \emph{YourGitHubUsername} to your GitHub username) and run the
  update script \texttt{./scripts/update.sh} (replace
  \emph{YourGitHubUsername} with your GitHub username):
  \begin{minted}{bash}
    cd  assignments-2020-YourGitHubUsername
    bash ./scripts/update.sh
  \end{minted}
  It should create three subdirectories\footnote{If the script fails,
    file an issue in the
    \href{https://github.com/ASU-CompMethodsPhysics-PHY494/PHY494-assignments-skeleton/issues}{Issue
      Tracker for PHY494-assignments-skeleton} and just create the
    directories manually.} \texttt{assignment\_\anum/Submission},
  \texttt{assignment\_\anum/Grade}, and
  \texttt{assignment\_\anum/Work}.
\item You can try out code in the \texttt{assignment\_\anum/Work}
  directory but you don't have to use it if you don't want to. Your
  grade with comments will appear in
  \texttt{assignment\_\anum/Grade}.
\item Create your solution in
  \texttt{assignment\_\anum/Submission}. Use Git to \texttt{git
    add} files and \texttt{git commit} changes.

  You can create a PDF, a text file or Jupyter notebook inside the
  \texttt{assignment\_\anum/Submission} directory as well as Python
  code (if required). \textbf{Name your files \texttt{hw\anum.pdf} or
    \texttt{hw\anum.txt} or \texttt{hw\anum.ipynb}}, depending on how
  you format your work. Files with code (if requested) should be named
  exactly as required in the assignment.
\item When you are ready to submit your solution, do a final
  \texttt{git status} to check that you haven't forgotten anything,
  commit any uncommited changes, and \texttt{git push} to your GitHub
  repository. Check on \emph{your} GitHub repository web
  page\footnote{\texttt{https://github.com/ASU-CompMethodsPhysics-PHY494/assignments-2020-\emph{YourGitHubUsername}}}
  that your files were properly submitted.

  You can push more updates up until the deadline. Changes after the
  deadline will not be taken into account for grading.
\end{enumerate}
Homeworks must be legible and intelligible or may otherwise be
returned ungraded with 0 points.

This assignment contains \textbf{bonus problems}. A bonus problem is
optional. If you do it you get additional points that count towards
this homework's total, although you can't get more than the maximum
number of points. If you don't do it you can still get full
points. Bonus problems and bonus points are indicated with an asterisk
``*''.\footnote{There are currently no tests available for this homework.}

% If you implement the functions as specified you can run the tests in
% the file \texttt{Submission/test\_hw\anum.py} with \texttt{pytest}
% \begin{minted}{bash}
%   cd Submission
%   pytest test_hw07.py
% \end{minted}
% and all tests should pass. If you have errors, have a look at the
% output and try to figure out what is still not working. Having the
% tests pass is not a guarantee that you will get full points (but it is
% general a very good sign!). 

\subsection{The Lennard-Jones potential (8 points)}
\label{sec:LJ}

The \emph{Lennard-Jones potential} is widely used to simulate the
behavior of atoms and molecules. It models the interactions between
two uncharged atoms as
\begin{gather}
  \label{eq:LJ}
  V(r) = 4\epsilon\left[\left(\frac{\sigma}{r}\right)^{12} - \left(\frac{\sigma}{r}\right)^{6}\right].
\end{gather}
where $r$ is the distance between two atoms at positions $\vec{r}_{1}$
and $\vec{r}_{2}$
\begin{gather}
  \label{eq:distance}
  r = \sqrt{\vec{r}\cdot\vec{r}} \quad\text{with}\quad \vec{r} :=
  \vec{r}_{2} - \vec{r}_{1}
\end{gather}
(and $\vec{r}$ is the distance vector).  The LJ potential describes
the atomic van der Waals interactions. They consist of an attractive
term (the $-(\sigma/r)^{6}$ term) due to London dispersion forces
(spontaneous fluctuations in the electron clouds lead to attractive
fluctuating dipoles), and a strongly repulsive term (the
$+(\sigma/r)^{12}$), which crudely models the fact that atoms cannot
come arbitrarily close (due to the quantum mechanical Pauli exclusion
principle).

The LJ potential Eq.~\ref{eq:LJ} has two parameters: $\epsilon$ is an
energy and sets the energy scale and is related to the attractive
strength whereas $\sigma$ is a length and sets the length scale and
range of the interaction. Parameters for the noble gas \textbf{argon}
in the liquid state are

\begin{center}
  \begin{tabular}{ccc}
    \toprule
    parameter & value & unit\\
    \midrule
    $\epsilon$ & $1.654 \times 10^{-21}$ & J\\
    $\sigma$ & $3.41\times 10^{-10}$& m\\
    \bottomrule                        
  \end{tabular}
\end{center}
For numerical convenience, we want to use units close to 1 so for all
further numerical evaluations we use
\begin{center}
  \begin{tabular}{ccc}
    \toprule
    parameter & value & unit\\
    \midrule
    $\epsilon/k_{B}$ & $119.8$ & K\\
    $\sigma$ & $0.341$& nm\\
    \bottomrule                        
  \end{tabular}
\end{center}
Note that we express $\epsilon$ as a temperature by dividing with the
Boltzmann constant
$k_{B} = 1.3806488 \times
10^{-23}\,\text{J}\cdot\text{K}^{-1}$. Temperature is a direct measure
of energy and in this case it is convenient to use Kelvin as a measure
of energy instead of Joule. The length $\sigma$ is expressed in nano
meters ($1\,\text{nm} = 1\times 10^{-9}\,\text{m}$).

In the following, use Python (with \texttt{numpy}) for all numerical
calculations and \texttt{matplotlib} for all plotting.
\begin{enuma}
\item \label{li:plot} Plot the Lennard-Jones potential Eq.~\ref{eq:LJ}
  for argon in the range $0 < r \leq 4\sigma$ \points{3}
  \begin{itemize}
  \item Use a sufficiently large number of $r$ values for plotting so
    that all details of $V(r)$ are visible in the plot\footnote{For
      instance, \texttt{np.linspace(1e-15, 4*sigma, 100)}}. 
  \item Set the display limits of the energy ($y$) axis to range from
    $-120$~K to $+100$~K.\footnote{Generate the plot with \texttt{ax =
        plt.subplot(1, 1, 1)}, then plot with \texttt{ax.plot(...)} and
      set the limits with \texttt{ax.set\_ylim(-120, 100)}.} Your plot
    should clearly show the smallest value of $V_{\text{LJ}}$ and
    indicate the behavior for small and large $r$.
  \item Label your axes with the quantity plotted and the
    units.\footnote{For example, for the $x$-axis you may use
      \texttt{ax.set\_xlabel(r"distance \$r\$ (nm)")}.}
  \end{itemize}
\item \label{li:analytical} The \emph{force} along the direction of
  $\vec{r}$ is
  \begin{gather}
    \label{eq:force}
    F = -\frac{dV(r)}{dr}.
  \end{gather}
  Calculate the Lennard-Jones force $F(r)$
  \emph{analytically}.\footnote{$F(r)$ is the force that one particle
    exerts on the other particle when they are at a distance $r$.}

  Plot $F(r)$ over the same range of $r$ values as in
  \ref{li:plot}. Adjust the force-axis limits so that you can clearly
  show the minimum in $F(r)$. \points{5}
\item \BONUS Analytically calculate the general value $r_{\text{min}}$
  at which the Lennard-Jones potential Eq.~\ref{eq:LJ} obtains a
  minimum and calculate its minimal value
  $V_{\text{LJ}}(r_{\text{min}}) = \min_{r} V_{\text{LJ}}(r)$. First
  express your answers in terms of $\epsilon$ and $\sigma$ and then
  calculate the specific values for argon. \bonus{2}
\end{enuma}

\subsection{Molecular dynamics simulation of the Argon dimer (20 points)}
\label{sec:argonmd}

At sufficiently low temperature, two atoms of the noble gas argon are
weakly bound to each other and form a \emph{dimer}. The interaction
between two argon atoms with mass $m$ each can be approximated by the
Lennard-Jones potential (Eq.~\ref{eq:LJ}).

Simulate the dynamics of the weakly bound argon dimer
Ar\textsubscript{2} by solving Newton's equations of motion for the
interatomic distance $r$
\begin{gather}
  \label{eq:eom}
  \frac{d^2 r}{dt^2} = \mu^{-1} F_{LJ}(r).
\end{gather}
with the reduced mass $\mu = m/2$. (We can treat this as a 1-body, 1D
problem because we can reduce the 2-body problem to an effective
1-body problem with the reduced mass
\begin{gather}
  \label{eq:mu}
  \mu = \frac{m_{1} m_{2}}{m_{1} + m_{2}} = \frac{m^{2}}{2m} = \frac{1}{2}m
\end{gather}
and if we start with zero total angular momentum then the only motion
is along the connecting vector and the problem can be fully described
in in 1D).

Use as parameters for argon
\begin{itemize}
\item
  $\epsilon = 120\,\text{K} \cdot k_B =
  0.99774\,\text{kJ}\cdot\text{mol}^{-1}$ (use
  $\text{kJ}\cdot\text{mol}^{-1}$ as units, see below)
\item $\sigma = 0.341\,\text{nm}$ 
\item $m = 39.948\,\text{u}$ (in atomic mass units,
  $1\,\text{u} = 1.660539040 \times 10^{-27}\,\text{kg} =
  1\,\text{g}\cdot\text{mol}^{-1}$)
\end{itemize}
When using these units for energy, length, and mass then for
consistency the unit of time \emph{must} be pico seconds (ps,
$10^{-12}$ seconds).\footnote{Using the atomic mass unit in
  conjunction with nm/ps as unit of velocity directly yields energies
  in kJ/mol, i.e.
  $\frac{1}{2} m v^{2} = \frac{1}{2} \times 1\,\text{u} \times
  (1\,\text{nm}\cdot\text{ps}^{-1})^2 =
  1\,\text{kJ}\cdot\text{mol}^{-1}$.}

For the following problems you can use the provided module
\texttt{integrators.py}, which was developed in the lesson on ODEs,
without having to specifically note this. If you use the module,
include it in your submission. You may also modify the module and
include it as part of your solution.

\begin{enuma}
\item \label{li:euler} Use \emph{Euler's method} to simulate the dynamics for
  $0 \leq t \leq 10\,\text{ps}$ with a time step
  $\Delta t = 1\times 10^{-4}\,\text{ps}$. Use as initial conditions
  \begin{itemize}
  \item $r(t=0) = 0.38276$ (nm) 
  \item $v(t=0) = 0.18$ (nm/ps).
  \end{itemize}
  Include all your functions in in a file \texttt{problem2.py} (which
  may itself import \texttt{integrators.py} if necessary).
  \begin{enumi}
  \item Create a function \mintinline{python}{F_LJ(r)} that returns
    the force (Eq.~\ref{eq:force}) at distance $r$ due to the
    Lennard-Jones potential Eq.~\ref{eq:LJ} for Argon. The units of
    this problem should be used (nm, kJ/mol, kJ/(nm mol) for
    force). \points{2}
  \item Create a function \mintinline{python}{dimer_md(r0, v0, t_max,
      dt, mass)} that takes the initial distance of the atoms $r(t=0)$
    (in nm) and initial relative velocity $v(t=0)$ (in nm/ps) as well
    as the maximum time $t_{\text{max}}$, the integration timestep
    $\Delta t$, and the mass as input and returns a tuple consisting
    of \mintinline{python}{(t, y)}, i.e., a numpy array of all the
    times $t$ and a numpy array consisting of an array of the ODE
    standard form of the dependent variables
    $\mathbf{y}(t) = (r(t), v(t))$, i.e., the positition and velocity
    at each timestep. \points{10}
  \item Plot $r(t)$ and show the plot; include axes labels with
    units. \points{2}
  \item Describe briefly what kind of motion the argon dimer
    undergoes. \points{1}
  \end{enumi}
\item \label{li:error} Analyze the error in the total energy $E$ by
  computing the total energy from potential $U$ and kinetic energy
  $T_{\text{kin}}$.
  \begin{enumi}
  \item Create a function \mintinline{python}{U_LJ(r)} that calculates
    the value of the Lennard-Jones potential Eq.~\ref{eq:LJ} for argon
    in the units of this problem (i.e., input nm, output
    kJ/mol). \points{2}
  \item Plot (1) $E(t)$, $U(t)$, $T_{\text{kin}}(t)$ and (2) the
    decadic logarithm of the relative error in the total energy, i.e.,
    \begin{gather*}
      \log_{10}\left|\frac{E(t)}{E(t=0)} - 1 \right|.
    \end{gather*}
    Label the graph axes with appropriate units. \points{2}
  \item Comment on your graph. \points{1}
  \end{enumi}
\item \BONUS Perform the MD in \ref{li:euler} with the \emph{semi-implicit
    Euler} algorithm (create a function
  \mintinline{python}{dimer_md_semiimplicit_Euler()}) and perform the error analysis
  as in \ref{li:error}. \bonus{10}
\end{enuma}

\end{document}

%%% Local Variables: 
%%% mode: latex
%%% TeX-master: t
%%% End: 
