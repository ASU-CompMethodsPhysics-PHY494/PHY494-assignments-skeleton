%%% generic article type (pdf)latex file
%%% use together with Makefile

\documentclass[paper=letter]{scrartcl}
\usepackage{graphicx}
\usepackage{amsmath,amsfonts,amsthm}
\usepackage{eufrak}
\usepackage{mathabx}
\usepackage{url}
\usepackage[usenames,dvipsnames,svgnames,table]{xcolor}
\usepackage[colorlinks]{hyperref}
\hypersetup{
     colorlinks   = true,
     urlcolor     = blue,
     linkcolor    = red,
     citecolor    = black
}
\usepackage{enumitem}
\usepackage{booktabs}
\usepackage{cprotect}
\usepackage{minted}

%\usepackage{wrapfig}
%\usepackage{subfig}
%\usepackage[format=plain,labelsep=period,font=small,labelfont=bf]{caption}

%------------------------------------------------------------
% assignment
%
\newcommand{\anumber}{6}
%
%------------------------------------------------------------

% hyperref https://en.wikibooks.org/wiki/LaTeX/Hyperlinks#.5Chref
\urlstyle{same}

%% not working yet...
\newcounter{TotalPoints}
\newcounter{TotalBonus}

\newcommand{\BONUS}{\textsc{Bonus: }}
\newcommand{\bonus}[1]{\textbf{[bonus +#1*]}\stepcounter{TotalBonus}}
\newcommand{\points}[1]{\textbf{[#1 points]}\stepcounter{TotalPoints}}
\newenvironment{enuma}{\begin{enumerate}[label=(\alph*)]}{\end{enumerate}}
\newenvironment{enumi}{\begin{enumerate}[label=(\roman*)]}{\end{enumerate}}
\newenvironment{solution}{\par\noindent\P{} }{\ \qedsymbol}

\renewcommand{\vec}[1]{\ensuremath{\mathbf{#1}}}
\newcommand{\pd}[3][]{\left(\frac{\partial #2}{\partial #3}\right)_{#1}}

\newcommand{\anum}{0\anumber}


\begin{document}
%\maketitle

\setcounter{section}{\anumber}
\addtocounter{section}{-1}
\section{ --- PHY 494: Homework assignment (20 points total)}

\noindent Due Thursday, Feb 27, 2020, 11:59pm.

\noindent
%  \url{}
\fbox{\parbox{\linewidth}{Submission is to your \textbf{private
      GitHub repository}.}}
Read the following instructions carefully. Ask if anything is unclear.
\begin{enumerate}
\item \texttt{cd} into your assignment repository (change
  \emph{YourGitHubUsername} to your GitHub username) and run the
  update script \texttt{./scripts/update.sh} (replace
  \emph{YourGitHubUsername} with your GitHub username):
  \begin{minted}{bash}
    cd  assignments-2020-YourGitHubUsername
    bash ./scripts/update.sh
  \end{minted}
  It should create three subdirectories\footnote{If the script fails,
    file an issue in the
    \href{https://github.com/ASU-CompMethodsPhysics-PHY494/PHY494-assignments-skeleton/issues}{Issue
      Tracker for PHY494-assignments-skeleton} and just create the
    directories manually.} \texttt{assignment\_\anum/Submission},
  \texttt{assignment\_\anum/Grade}, and
  \texttt{assignment\_\anum/Work}.
\item You can try out code in the \texttt{assignment\_\anum/Work}
  directory but you don't have to use it if you don't want to. Your
  grade with comments will appear in
  \texttt{assignment\_\anum/Grade}.
\item Create your solution in
  \texttt{assignment\_\anum/Submission}. Use Git to \texttt{git
    add} files and \texttt{git commit} changes.

  You can create a PDF, a text file or Jupyter notebook inside the
  \texttt{assignment\_\anum/Submission} directory as well as Python
  code (if required). \textbf{Name your files \texttt{hw\anum.pdf} or
    \texttt{hw\anum.txt} or \texttt{hw\anum.ipynb}}, depending on how
  you format your work. Files with code (if requested) should be named
  exactly as required in the assignment.
\item When you are ready to submit your solution, do a final
  \texttt{git status} to check that you haven't forgotten anything,
  commit any uncommited changes, and \texttt{git push} to your GitHub
  repository. Check on \emph{your} GitHub repository web
  page\footnote{\texttt{https://github.com/ASU-CompMethodsPhysics-PHY494/assignments-2020-\emph{YourGitHubUsername}}}
  that your files were properly submitted.

  You can push more updates up until the deadline. Changes after the
  deadline will not be taken into account for grading.
\end{enumerate}
Homeworks must be legible and intelligible or may otherwise be
returned ungraded with 0 points.

This assignment contains \textbf{bonus problems}. A bonus problem is
optional. If you do it you get additional points that count towards
this homework's total, although you can't get more than the maximum
number of points. If you don't do it you can still get full
points. Bonus problems and bonus points are indicated with an asterisk
``*''.

For problem \ref{sec:exp}: If you implement the
function as specified you can run the tests in the file
\texttt{Submission/test\_hw0\anum.py} with \texttt{pytest}
\begin{minted}{bash}
  cd Submission
  pytest test_hw06.py
\end{minted}
and all tests should pass. If you have errors, have a look at the
output and try to figure out what is still not working. Having the
tests pass is not a guarantee that you will get full points (but it is
general a very good sign!). 

\subsection{BONUS: Discussion of the errors of finite difference operators
  (+7* bonus points)}
\label{sec:fderrors}

In lesson
\href{https://asu-compmethodsphysics-phy494.github.io/ASU-PHY494/2020/02/25/09_Differentiation/}{09 Differentiation} you plotted the absolute error
$|E_{h}(t)| = |D_{h}\,\cos t - \cos t|$ for three different algorithms
for the finite difference operator $D_{h}$ with step size $h$ (forward
difference, central difference, and extended difference) and for three
different values $t = 0.1, 1, 100$.

Complete all calculations and compare your plots to the ones shown at
the end of the notebook
\href{https://github.com/ASU-CompMethodsPhysics-PHY494/PHY494-resources/blob/master/09_differentiation/09-differentiation.ipynb}{09-differentiation.ipynb}. Discuss
the following questions (write them in a simple text file
\texttt{problem1.txt} or you can also submit a notebook
\texttt{problem1.ipynb} but make clear where you answer the questions
below):

\begin{enuma}
\item Which algorithm produces the most accurate\footnote{The
    \emph{accuracy} is measured by the deviation from the exact
    result, i.e. the lower $|E|$ the more accurate. \emph{Precision}
    measures how well a number is defined and is essentially the
    machine precision in our case.} result? Give order-of-magnitude
  estimates for each one, based on your data. \bonus{2}
\item \label{li:graphs} Describe the general shape and features of
  your graphs. Explain why you see increases and decreases in accuracy
  with varying $h$. \bonus{2}
\item What is the best value of $h$ in each case? How does the best
  value of $h$ change with the algorithm? Explain the observed
  behavior in the light of the answer to your answer \ref{li:graphs}. 
  \bonus{3}
\end{enuma}

\subsection{Exponential function (20 points)}
\label{sec:exp}

The exponential function has the series expansion
\begin{gather}
  \exp x = \sum_{n=0}^{+\infty} \frac{x^n}{n!}\label{eq:series}  
\end{gather}

An algorithm to compute $\exp x$ makes use of the iterative
solution\footnote{Similar to the iterative solution for the $\sin x$
  function that was discussed in
  \href{https://asu-compmethodsphysics-phy494.github.io/ASU-PHY494/2020/02/12/07_Numbers/}{Lecture
    07}.}
\begin{align}
  a_n &= \frac{x^n}{n!} \label{eq:an}\\
  a_{n+1} &= a_n q_{n+1} = \frac{x^{n+1}}{(n+1)!} = \frac{x^{n}}{n!} \frac{x}{n+1} \label{eq:an1}\\
  q_n &= \frac{x}{n+1} \label{eq:qn}
\end{align}

\begin{enuma}
\item Create a function \texttt{exp\_series(x, eps=1e-15)} in a file
  \texttt{problem2.py} that computes the $\exp(x)$ function based on the
  series expansion Eq.~\ref{eq:series} and the iterative solution
  Eq.~\ref{eq:an1}--\ref{eq:qn}. The function should only return the
  value of $\exp(x)$.

  The function should take an argument \texttt{x} and optional
  convergence criterion \texttt{eps} with default \texttt{1e-15}.

  The iteration should stop when the \textbf{convergence criterion}
  \begin{equation}
    \left|\frac{a_{N}}{\sum_{n=0}^{N} a_{n}}\right| \le \epsilon
    \label{eq:convergence}  
  \end{equation}
  is fulfilled. \points{12}
  
\item Show results for $\epsilon = 10^{-15}$ and
  $x = -9.2103437, 0, 1, 100$  \points{4}
  
\item Show results for $\epsilon = 10^{-4}$ and
  $x = -9.2103437, 0, 1, 100$ \points{4}
\end{enuma}

\subsection{BONUS CHALLENGE: Stable sum (+10* points)}
\label{sec:stablesum}

\emph{This problem is a bonus problem that challenges you to do your
  own research and come up with a working solution.}

The normal Python \mintinline{python}{sum()} or
\mintinline{python}{numpy.sum()} functions can easily loose
precision. As an example, try calculating the sum over the sequence
\begin{gather*}
  S_{M} = \{\underbrace{1, 10^{100}, 1, -10^{100}}_{\text{repeated $M$ times}}, \dots, 1, 10^{100}, 1, -10^{100}\}
\end{gather*}
which is
\begin{gather}
  \label{eq:sum}
  \sum_{x \in S_{M}}x = 2M.
\end{gather}
In Python you can generate $S_{M}$ with the ``list multiplication'' operation
\begin{minted}{python}
M = 10000
SM = [1, 1e100, 1, -1e100] * M
\end{minted}
(which generates a new list that consists of the concatenation of
\mintinline{python}{M} copies of the original list).

\begin{enuma}
\item Convince yourself that \mintinline{python}{sum()} and
  \mintinline{python}{numpy.sum()} give the wrong answer for
  Eq.~\ref{eq:sum} and $M=10000$. Show the results. \bonus{1}
\item Show that \mintinline{python}{math.fsum()} gives the correct
  answer. \bonus{1}
\item Implement an algorithm in a function
  \mintinline{python}{stable_sum()} in a file \texttt{problem3.py}
  that takes a sequence of numbers as input and calculates the sum
  Eq.~\ref{eq:sum} correctly. You cannot use
  \mintinline{python}{math.fsum()}. If you implement an algorithm from
  the literature, cite your sources. Show the output of your algorithm
  for the sum for $M=10,000$ and $M=10^{6}$. \bonus{8}
\end{enuma}

\end{document}

%%% Local Variables: 
%%% mode: latex
%%% TeX-master: t
%%% End: 
