%%% generic article type (pdf)latex file
%%% use together with Makefile

\documentclass[letterpaper]{scrartcl}
\usepackage{graphicx}
\usepackage{amsmath,amsfonts,amsthm}
\usepackage{eufrak}
\usepackage{mathabx}
\usepackage{url}
\usepackage{framed}
\usepackage[usenames,dvipsnames,svgnames,table]{xcolor}
\usepackage[colorlinks]{hyperref}
\hypersetup{
     colorlinks   = true,
     urlcolor     = blue,
     linkcolor    = red,
     citecolor    = black
}
\usepackage{enumitem}
\usepackage{booktabs}
\usepackage{cprotect}
\usepackage{minted}

%\usepackage{wrapfig}
%\usepackage{subfig}
\usepackage[format=plain,labelsep=period,font=small,labelfont=bf]{caption}

%------------------------------------------------------------
% assignment
%
\newcommand{\anumber}{1}
%
%------------------------------------------------------------

% hyperref https://en.wikibooks.org/wiki/LaTeX/Hyperlinks#.5Chref
\urlstyle{same}

%% not working yet...
\newcounter{TotalPoints}
\newcounter{TotalBonus}

\newcommand{\BONUS}{\textsc{Bonus: }}
\newcommand{\bonus}[1]{\textbf{[bonus +#1*]}\stepcounter{TotalBonus}}
\newcommand{\points}[1]{\textbf{[#1 points]}\stepcounter{TotalPoints}}
\newenvironment{enuma}{\begin{enumerate}[label=(\alph*)]}{\end{enumerate}}
\newenvironment{enumi}{\begin{enumerate}[label=(\roman*)]}{\end{enumerate}}
\newenvironment{solution}{\par\noindent\P{} }{\ \qedsymbol}

\renewcommand{\vec}[1]{\ensuremath{\mathbf{#1}}}
\newcommand{\pd}[3][]{\left(\frac{\partial #2}{\partial #3}\right)_{#1}}

\newcommand{\anum}{\anumber}

\title{{\Large Project \anumber: A Dilemma!}}

\author{{\sffamily\large Project \emph{Comp. Methods Physics} ASU
    PHY494 (2018)}%
  \thanks{Current version of this document: \today. See
    Appendix~\protect\ref{sec:history} for a list of changes since v1
    from Feb 06, 2018.}}

\date{{\sffamily\large February 8, 2018 -- Feb 22, 2018}}


\begin{document}
\maketitle

\paragraph{Abstract}

You will investigate the decisions that a driver can make when braking
for a traffic light that switches from green to yellow by writing code
in Python to model the car's behavior for different decisions and
parameters. You will write a short report to communicate, discuss and
summarize your reasoning and your results.

\begin{framed}
  \noindent
  Due Thursday, February 22, 2018, 11:59pm.
  \begin{itemize}
  \item Each student works on their own project.
  \item \textbf{Admissible Collaboration:} Students are allowed to
    talk to other students in the class about the project and exchange
    ideas and tips. However, sharing/copying reports or full code
    solutions is not allowed. \textbf{Help from other students must be
      acknowledged in an Acknowledgments section}.  Direct help from
    outside the class is not allowed (except instructor/TA), e.g.,
    you cannot ask for solutions (online or in person) but you can use
    books and the internet to solve problems.
  \item Each student should commit their \emph{own} report (see
    Section~\ref{sec:report}) in \textbf{PDF} format to their own
    \textbf{GitHub repository}; alternatively, combining report and
    code in a Jupyter notebook is also possible as long as the
    notebook can be read like a report (i.e., not just bullet points
    or short comments).
  \item Each student should commit and push \textbf{all code} (see
    Section \ref{sec:code}) that is required to reproduce the results
    in the report to their own \textbf{GitHub repository}. Include a
    text file \textbf{\texttt{README.txt}} that describes the commands
    to run calculations. The code must run in the standard
    anaconda-based environment used for the class. If it is a Jupyter
    notebook then it should be possible to \emph{Kernel $\rightarrow$
      Restart \& Run All} and to produce all the required figures and
    output.
  \end{itemize}
  Grading will take the following into consideration:
  \begin{itemize}
  \item code runs and produces correct output
  \item report clearly and succinctly describes the question,
    approach, and results and contains sufficient evidence that the
    requirements (see below) have been met
  \item thorough attribution of code and help
  \item any additional work that you want to include in an appendix to
    the report or additional simulations for the main report will be
    treated as bonus material
  \end{itemize}
\end{framed}

\section{Submission instructions}

\noindent
%  \url{}
\fbox{\parbox{\linewidth}{Submission is now to your \textbf{private
      GitHub repository}. Follow the link provided to you by the
    instructor in order for the repository to be set up: It will have
    the name
    \emph{ASU-CompMethodsPhysics-PHY494/assignments-2018-\emph{YourGitHubUsername}}
    and will only be visible to you and the instructor/TA. Follow the
    instructions below to submit this (and all future) homework.}}
Read the following instructions carefully. Ask if anything is unclear.
\begin{enumerate}
\item \texttt{git clone} your assignment repository (change
  \emph{YourGitHubUsername} to your GitHub user name)
  \begin{minted}[autogobble]{bash}
    repo="assignments-2018-YourGitHubUsername.git" 
    git clone https://github.com/ASU-CompMethodsPhysics-PHY494/${repo}
  \end{minted}
  or, if you already have done so, \mintinline{bash}{git pull} from
  within your assignments directory.
\item run the script
  \texttt{./scripts/update.sh} (replace \emph{YourGitHubUsername} with
  your GitHub user name):
  \begin{minted}[autogobble]{bash}
    cd ${repo} 
    bash ./scripts/update.sh
  \end{minted}
  It should create three sub-directories\footnote{If the script fails,
    file an issue in the
    \href{https://github.com/ASU-CompMethodsPhysics-PHY494/PHY494-assignments-skeleton/issues}{Issue
      Tracker for PHY494-assignments-skeleton} and just create the
    directories manually.} \texttt{project\_\anum/Submission},
  \texttt{project\_\anum/Grade}, and
  \texttt{project\_\anum/Work}.
\item You can try out code in the \texttt{project\_\anum/Work}
  directory but you don't have to use it if you don't want to. Your
  grade with comments will appear in
  \texttt{project\_\anum/Grade}.
\item Create your solution in
  \texttt{project\_\anum/Submission}. Use Git to \texttt{git
    add} files and \texttt{git commit} changes.

  You can create a PDF file or Jupyter notebook inside the
  \texttt{project\_\anum/Submission} directory as well as Python
  code (if required). \textbf{Name your files \texttt{project\anum.pdf} or
    or \texttt{project\anum.ipynb}}, depending on how
  you format your work. Files with code (if requested) should be named
  exactly as required in the assignment.
\item When you are ready to submit your solution, do a final
  \texttt{git status} to check that you haven't forgotten anything,
  commit any uncommitted changes, and \texttt{git push} to your GitHub
  repository. Check on \emph{your} GitHub repository web
  page\footnote{\texttt{https://github.com/ASU-CompMethodsPhysics-PHY494/assignments-2018-\emph{YourGitHubUsername}}}
  that your files were properly submitted.

  You can push more updates up until the deadline. Changes after the
  deadline will not be taken into account for grading.
\end{enumerate}
Work must be legible and intelligible or may otherwise be
returned ungraded with 0 points.

\section{Problem description}
\label{sec:problem}

A driver in a car approaches an intersection with constant speed
$v_0 = 55 \text{km/h}$. She sees the traffic light switch from green
to yellow when she is at a distance $x_0$ from the intersection. She
has to decide
\begin{itemize}
\item to brake (with deceleration $a = -3\,\text{m/s}$) or
\item to continue driving with constant $v_0$.\footnote{In this
    problem the driver does not have the (legal) option to accelerate
    because she is already driving at the speed limit.}
\end{itemize}
Any reaction is delayed by the driver's reaction time
$\delta = 0.8\,\text{s}$. The yellow interval (time between green and
red) is $\tau = 3\,\text{s}$ and the width of the intersection is
$W = 45\,\text{m}$. You can ignore the length of the car. Take the
entrance of the intersection and the position of the traffic light to
be at $x=0$ (see Figure~\ref{fig:setup}).

\begin{figure}[btp]
  \centering
  \includegraphics[width=0.6\linewidth]{figs/p1_setup}
  \caption{Geometry of the traffic situation. $x_{0}$ is the position
    of the car at time $t=0$ when the traffic light at $x=0$ switches
    to yellow. The width of the intersection is $W$. If $x_{0} \ge
    x_{0}^{A}$ then the car will be able to cross the intersection
    before the light switches to red; if $x_{0} \le x_{0}^{B}$ then
    the car will be able to brake before the traffic light. The gray
    zone is the ``\emph{dilemma zone}''.}
  \label{fig:setup}
\end{figure}

Your task will be to determine if there are situations in which the
driver will not be able to adhere to the traffic code, i.e., if she
will either run a red light or not clear the intersection before the
traffic light has switched to red. In particular, you should find out
if there is a starting position ($x_{0}^{B}$ in
Figure~\ref{fig:setup}) before which braking is safe (i.e., the car
will stop before the traffic light) and if there is a starting
position ($x_{0}^{A}$) beyond which crossing the intersection is safe
(i.e., the car will cross the intersection before the light has
switched from yellow to red).

Under certain conditions there can exist a range of values
$x_{0}^{B} < x_{0} < x_{0}^{A}$ where neither the decision to drive
across the intersection nor to brake will be successful and therefore
lead to a dangerous traffic situation. This range of values is called
the \emph{dilemma zone} (see the gray area in
Figure~\ref{fig:setup}). The size of the dilemma zone is
\begin{gather}
  \label{eq:dilemma}
  s := x_{0}^{A} - x_{0}^{B}
\end{gather}
and the dilemma zone exists when $s > 0$.

\section{Report}
\label{sec:report}

Write a report in which you address the tasks in Section
\ref{sec:tasks} below. The report should contain all results (figures,
tables, equations). It must contain a title, author's name, sections
\emph{Background} (problem description, definitions), \emph{Results
  and Discussion} (description and interpretation of results), and
\emph{Summary} (short summary of the main results). If you had any
form of outside help you must describe it in an \emph{Acknowledgments}
section. If you use code or material from elsewhere you \emph{must
  cite the source} (add a \emph{References} section). Any bonus
material can be shown in an optional \emph{Appendix}.

The report must be written in full sentences and read as a coherent
piece of work. Figures must have legends, labels, and captions. Type
set in an 11pt font with single line spacing (captions, labels, legend
may have smaller font sizes but must still be legible) and leave at
least 1~in margins. 

Overall, a length of about four pages is expected; the report should
not be less than three or more than six pages long.

\section{Code}
\label{sec:code}

For all numerical calculations use Python 3.x. You may use any of the
Python packages that are part of the Anaconda 3 distribution such as
\texttt{numpy} and \texttt{matplotlib}. 

Include all the code that is needed to generate the results shown in
your report. This can consist of Python programs, modules, a Jupyter
notebook, or a mixture thereof. Include a separate file
\texttt{README.txt} that explains how to run your code in order to
generate the results in your report. Your code must run in order for
you to be awarded full marks.


\section{Tasks}
\label{sec:tasks}
Address the following tasks in your report:

\begin{enuma}
\item Write down the equations of motions of the car, namely $v(t)$
  and $x(t)$, depending on $x_0$ for both the decision to brake or to
  continue driving.\footnote{The Heaviside step function
    Eq.~\ref{eq:heaviside} and its Python implementation in Appendix
    \ref{sec:heaviside} might be useful.}\label{li:eom}

\item Use your equations of motions from \ref{li:eom} to define Python
  functions to compute the velocity $v(t)$ and position $x(t)$ for any
  $t$, depending on $x_0$ and for both the decision to brake or to
  continue driving.

  Plot $v(t)$ and $x(t)$ for $0 \le t \le 7\,\text{s}$ in time steps
  of $\Delta t = 0.1\,\text{s}$ for $x_{0} = -30\,\text{m}$ (which
  should look similar to Figure~\ref{fig:timeseries}) as well as
  $x_{0} = -70\,\text{m}$ and $x_{0} = -0.5\,\text{m}$. Plot graphs
  for \emph{driving} and \emph{braking} in the same figure. Indicate
  the time $\tau$ at which the traffic light switches to red and the
  extent of the intersection $W$ in your figures.\footnote{See
    Appendix \ref{sec:plot} for Python code to plot a dashed vertical
    line and a gray rectangle, similar to those shown in
    Figure~\protect\ref{fig:timeseries}.} For each figure explain
  whether a safe or a dangerous situation is occuring (see also
  \ref{li:classification}).
   
  \begin{figure}
    \centering
    \includegraphics[width=0.8\linewidth]{figs/pos_vel_x0-30m}
    \caption{Example plot for time series $x(t)$ and $v(t)$ for
      $x_0 = -30\,\text{m}$. The red dashed line indicates time $\tau$
      and the shaded gray area indicates the extent of the crossing
      $W$.}
    \label{fig:timeseries}
  \end{figure}

\item \label{li:classification} Classify outcomes as either
  \emph{safe} or \emph{dangerous}, depending on $x_{0}$: For a given
  $x_{0}$, calculate the time series of the velocities
  $v_{\text{drive}}(t)$ and positions $x_{\text{drive}}(t)$ when the
  decision is made to attempt to \emph{drive} across the
  intersection. Also calculate the time series $v_{\text{break}}(t)$
  and $x_{\text{break}}(t)$ for the decision to brake. Use a range of
  time points $0 \le t_{i} \le t_{\text{max}}$ and
  $\Delta t = 0.1\,\text{s}$. You need choose $t_{\text{max}}$
  sufficiently large to see the car to come to a stop or to cross the
  intersection.

  Analyze the position trajectories $x(t)$ for driving/braking: The
  outcome is safe if
  \begin{description}
  \item[driving] $x>W$ for $t>\tau$ 
  \item[braking] $x<0$ for all $t$
  \end{description}
  Otherwise it is dangerous. For a given $x_{0}$, the outcome is safe
  if there is at least one safe course of action.

  Determine if the situation is safe for $x_{0} = -70\,\text{m}$,
  $-30\,\text{m}$, and $-0.5\,\text{m}$.  

 \item Map out the zone of values $x_{0}$ that lead to dangerous
  situations, i.e., that do not allow the driver to either brake or
  complete the crossing of the intersection in time.

  Run the calculations for $x_0$ ranging from $-100\,\text{m}$ to the
  entrance of the intersection $x_0=0$ in $0.5\,\text{m}$ intervals
  and determine for each $x_{0}$ if allows for a safe decision or
  not. 

  \begin{enumi}
  \item Plot the classification (e.g., \texttt{True} for \emph{safe},
    \texttt{False} for \emph{dangerous}) against $x_{0}$. Show and
    discuss your plot.
  \item Determine the numerical values of the dilemma zone
    $x_{0}^{A}$, $x_{0}^{B}$, and $s$ (Eq.~\ref{eq:dilemma}) from your
    data.
  \end{enumi}

\item Solve the whole problem analytically:
  \begin{enumi}
  \item Find general expressions for $x_{0}^{A}$, $x_{0}^{B}$, and
    $s$, which should only depend on $x_{0}$, $v_{0}$, $\delta$,
    $\tau$, and $W$.
  \item Compute $x_{0}^{A}$, $x_{0}^{B}$, and $s$ for the given
    parameters of the problem and compare to your simulation results.
  \end{enumi}

\item \BONUS\footnote{You do not have to work on this problem and you
    can still get 100\% of the points on the whole project but if you
    include results for the \BONUS problem then you can get additional
    bonus points.}  Determine the size of the dilemma zone as a
  function of the initial speed, $s(v_{0})$, and plot it for
  $20\,\text{km/h} \le v_{0} \le 100\,\text{km/h}$. In a second plot
  show $x_{0}^{A}(v_{0})$ and $x_{0}^{B}(v_{0})$ for the same range of
  $v_{0}$. You may solve this problem either numerically or
  analytically.
\end{enuma}

\appendix{}

\section{Heaviside step function}
\label{sec:heaviside}

The Heaviside step function
\begin{gather}
  \label{eq:heaviside}
  \Theta(x) =
  \begin{cases}
    1,& \quad x > 0\\
    \frac{1}{2},& \quad x = 0\\
    0,& \quad x < 0
  \end{cases}
\end{gather}
can be computed with the Python function
\begin{minted}[linenos,breaklines]{python}
import numpy as np

def heaviside(x):
    """Heaviside step function.
    
    Arguments
    ---------
    x : scalar or array
    
    Returns
    -------
    float64 scalar or array
    
    From http://stackoverflow.com/a/15122658/334357
    """
    return 0.5 * (np.sign(x) + 1)  
\end{minted}
It has the advantage that it functions as a NumPy ufunc, i.e., it
works equally well with scalar and array input.

\section{Plotting}
\label{sec:plot}

The following code can be used (perhaps with modifications) to add a
vertical dashed line to a plot and to plot a gray filled area:
\begin{minted}[linenos,breaklines]{python}
import matplotlib.pyplot as plt

# plot dashed red line at tau
plt.plot([tau, tau], [x0, 1.05*W], "--", color="red", lw=1)

# plot gray area to indicate intersection
plt.fill_between([0, 7], [W, W], color="black", alpha=0.3)
\end{minted}

\section{History}
\label{sec:history}

Changes and updates to this document.
\begin{description}
\item[2018-02-06] initial version
\end{description}

\end{document}

%%% Local Variables: 
%%% mode: latex
%%% TeX-master: t
%%% End: 
