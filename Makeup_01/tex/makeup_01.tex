%%% generic article type (pdf)latex file
%%% use together with Makefile

\documentclass[letterpaper]{scrartcl}
\usepackage{graphicx}
\usepackage{amsmath,amsfonts,amsthm,amsbsy}
\usepackage{eufrak}
\usepackage{mathabx}
\usepackage{courier}
\usepackage{url}
\usepackage{color}
\usepackage[usenames,dvipsnames,svgnames,table]{xcolor}
\usepackage{hyperref}
\hypersetup{
     colorlinks   = true,
     urlcolor     = blue,
     linkcolor    = red,
     citecolor    = black
}


\usepackage{enumitem}
\usepackage{booktabs}
\usepackage{cprotect}
\usepackage{listings}

\lstset{
  language=python, 
  basicstyle=\footnotesize\ttfamily
}

%\usepackage{wrapfig}
%\usepackage{subfig}
%\usepackage[format=plain,labelsep=period,font=small,labelfont=bf]{caption}

%------------------------------------------------------------
% makeup
%
\newcommand{\anumber}{1}
%
%------------------------------------------------------------

% hyperref https://en.wikibooks.org/wiki/LaTeX/Hyperlinks#.5Chref
\urlstyle{same}


%% not working yet...
\newcounter{TotalPoints}
\newcounter{TotalBonus}

\newcommand{\BONUS}{\textsc{Bonus: }}
\newcommand{\bonus}[1]{\textbf{[bonus +#1*]}\stepcounter{TotalBonus}}
\newcommand{\points}[1]{\textbf{[#1 points]}\stepcounter{TotalPoints}}
\newenvironment{enuma}{\begin{enumerate}[label=(\alph*)]}{\end{enumerate}}
\newenvironment{enumi}{\begin{enumerate}[label=(\roman*)]}{\end{enumerate}}
\newenvironment{solution}{\par\noindent\P{} }{\ \qedsymbol}

\renewcommand{\vec}[1]{\ensuremath{\mathbf{#1}}}
\newcommand{\pd}[3][]{\left(\frac{\partial #2}{\partial #3}\right)_{#1}}




\begin{document}
%\maketitle

\setcounter{section}{\anumber}
\addtocounter{section}{-1}
\section{ --- PHY 494: Makeup assignment \anumber{} (15 points total)}

\noindent Due Thursday, April 5, 2018, 11:59pm.

\noindent This is a \textbf{optional Makeup Assignment}. If you choose
to hand it in by the deadline, it will be graded like a normal
homework assignment. If its grade is better than your worst homework
grade then it will replace that grade.

\noindent
\fbox{\parbox{\linewidth}{Submission is to your \textbf{private
      GitHub repository}.}}

Enter the repository and run the script
\texttt{scripts/update.sh} (replace \emph{YourGitHubUsername} with
your GitHub username):
\begin{lstlisting}[language=sh]
cd assignments-2018-YourGitHubUsername
bash ./scripts/update.sh 
\end{lstlisting} 
It should create three subdirectories\footnote{If the script fails,
  file an issue in the
  \href{https://github.com/ASU-CompMethodsPhysics-PHY494/PHY494-assignments-skeleton/issues}{Issue
    Tracker for PHY494-assignments-skeleton} and just create the
  directories manually.} \texttt{makeup\_0\anumber{}/Submission},
\texttt{makeup\_0\anumber{}/Grade}, and
\texttt{makeup\_0\anumber{}/Work} and also pull in the PDF of the
makeup and an additional file.

To submit your makeup assignment, commit and push Python code inside
the \texttt{makeup\_0\anumber{}/Submission} directory. \emph{Commit
  any other additional files exactly as required in the problems.}

Failure to adhere to the following requirements may lead to homework
being returned ungraded with 0 points for the problem.
\begin{itemize}
\item Only submit code.
\item All code should be in a file \texttt{makeup01.py}.
\item Code will be tested against the unit tests in
  \texttt{test\_makeup01.py}. The grade will be approximately
  proportional to the number of tests that pass successfully so your
  code \emph{must} be able run under the tests. (Failing tests for the
  Bonus problem can be ignored.)
\end{itemize}


\subsection{ODE with Scipy (15 points)}
\label{sec:ode}

In this problem you should learn how to use existing functions in a
library, namely in the \texttt{scipy} package. Part of the problem is
to find and read the documentation for the function and figure out how
to use it.

Use the Scipy function \texttt{scipy.integrate.odeint()} to solve the
following ordinary differential equation 
\begin{gather}
  \label{eq:schrodinger}
  -\frac{1}{2}\frac{d^{2}\psi_{n}(x)}{dx^{2}} + \left[\frac{1}{2}
    x^{2} - E_{n}\right] \psi_{n}(x) = 0\\
  \label{eq:initial}
  \psi_{n}(0) = 1;\quad \frac{d\psi_{n}(0)}{dx} = 0\\
  \label{eq:energies}
  E_{n} = n + \frac{1}{2}, \quad n = 0, 2, 4, 6, \dots
\end{gather}
for the real function $\psi_{n}(x) $\footnote{If you have done Quantum
  Mechanics 1 (PHY 314) then you should recognize it as the
  Schr{\"o}dinger equation for the simple harmonic oscillator.} and the
three values $n=0$, $n=2$, and $n=8$.

The code should contain the following functions:
\begin{enumerate}
\item \textbf{ode\_qmhosc()} solves the ODE Eq.~\ref{eq:schrodinger}:
\begin{lstlisting}
psi = ode_qmhosc(x, psi0, dpsidx0, n=n)
\end{lstlisting}
  that takes as arguments
  \begin{itemize}
  \item all the values $x$ at which the solution should be evaluated
    (the first value \emph{must} be the one for which the initial
    conditions are given, i.e., $x=0$ for this problem),
  \item as initial conditions the function value and first derivative
    at $x=0$ (Eq.~\ref{eq:initial}), and
  \item the value of $n$ that determines $E_{n}$
    (Eq.~\ref{eq:energies}).
  \end{itemize}
  It should return the function values $\psi_{n}(x)$ as a numpy array.
\item \textbf{f\_qmhosc()} is needed to transform the ODE
  Eq.~\ref{eq:schrodinger} into standard form so that it can be solved
  with \texttt{scipy.integrate.odeint()}.
\begin{lstlisting}
f_qmhosc(y, t, E=0)
\end{lstlisting}
  should produce the \emph{ODE standard form}\footnote{Remember that
    the ODE standard form is
    \begin{gather*}
      \frac{d\vec{y}}{dt} = \vec{f}
    \end{gather*}
    and in the case of Eq.~\ref{eq:schrodinger}, we would have
    $t = x$, $y_{0} = \psi_{n}(x)$, and
    $y_{1} = \psi_{n}'(x) \equiv \frac{d\psi_{n}}{dx}$.} of the derivative
  vector $\vec{f}$ when provided with
  \begin{itemize}
  \item the current values of $\vec{y}$
  \item the current value of $t$
  \item and the parameter $E_{n}$ (see Eq.~\ref{eq:schrodinger}).
  \end{itemize}
\end{enumerate}

\begin{enuma}
  \item Numerically solve Eq.~\ref{eq:schrodinger} on a lattice with
    $\Delta x = 0.01$ for $0 \leq x < 6$ for
    \begin{itemize}
    \item $n=0$
    \item $n=2$
    \item $n=8$
    \end{itemize}
    \begin{enumi}
    \item The tests in \texttt{test\_makeup01.py} should all pass to
      show that you correctly implemented a solution.  \points{12}
    \item Plot all your solutions in one figure and include the figure
      as a PDF named \texttt{qmhosc.pdf}. \points{3}
    \end{enumi}
  \item \BONUS Correct physical\footnote{Physical solutions for the
      wavefunction $\psi_{n}(x)$ have to be normalizable, i.e., they
      have to decay to zero for $x \rightarrow \pm\infty$. The
      numerical solutions only fulfill this requirement for small $x$
      and even for moderately large $x$ they start to diverge. It
      would be better to use specialized algorithms that have the
      normalizability requirement built in.} solutions for
    Eq.~\ref{eq:schrodinger} are \emph{symmetric}
    ($\psi(x) = \psi(-x)$) for even $n$ and \emph{antisymmetric}
    ($\psi(x) = -\psi(-x)$) for odd $n$. This implies that the
    \emph{initial conditions} for $n=1, 3, 5, \dots$ are different
    from the ones for even $n$ (Eq.~\ref{eq:initial}). Choose
    appropriate initial conditions\footnote{Choose $\psi(0) \geq 0$,
      $\frac{d\psi(0)}{dx} \geq 0$ and either use values of 0 or 1
      because this is the convention employed in the tests.} and write
    a function
\begin{lstlisting}
phi0, dphidx0 = initial_values_qmhosc(n)
\end{lstlisting}
    that produces the correct initial values depending on the
    value of $n$.

    Using your initial values, solve Eq.~\ref{eq:schrodinger} for
    $n=3$ and plot the solution (include a PDF named
    \texttt{qmhosc\_odd.pdf}). \bonus{4}
\end{enuma}
% http://docs.scipy.org/doc/scipy/reference/tutorial/integrate.html#ordinary-differential-equations-odeint

\end{document}

%%% Local Variables: 
%%% mode: latex
%%% TeX-master: t
%%% End: 
