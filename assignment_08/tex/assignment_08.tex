%%% generic article type (pdf)latex file
%%% use together with Makefile

\documentclass[letterpaper]{scrartcl}
\usepackage{graphicx}
\usepackage{amsmath,amsfonts,amsthm,amsbsy}
\usepackage{eufrak}
\usepackage{mathabx}
\usepackage{courier}
\usepackage{url}
\usepackage{color}
\usepackage[usenames,dvipsnames,svgnames,table]{xcolor}
\usepackage{hyperref}
\hypersetup{
     colorlinks   = true,
     urlcolor     = blue,
     linkcolor    = red,
     citecolor    = black
}


\usepackage{enumitem}
\usepackage{booktabs}
\usepackage{cprotect}
\usepackage{minted}

%\usepackage{wrapfig}
%\usepackage{subfig}
%\usepackage[format=plain,labelsep=period,font=small,labelfont=bf]{caption}

%------------------------------------------------------------
% assignment
%
\newcommand{\anumber}{8}
%
%------------------------------------------------------------
\newcommand{\anum}{0\anumber}

% hyperref https://en.wikibooks.org/wiki/LaTeX/Hyperlinks#.5Chref
\urlstyle{same}


%% not working yet...
\newcounter{TotalPoints}
\newcounter{TotalBonus}

\newcommand{\BONUS}{\textsc{Bonus: }}
\newcommand{\bonus}[1]{\textbf{[bonus +#1*]}\stepcounter{TotalBonus}}
\newcommand{\points}[1]{\textbf{[#1 points]}\stepcounter{TotalPoints}}
\newenvironment{enuma}{\begin{enumerate}[label=(\alph*)]}{\end{enumerate}}
\newenvironment{enumi}{\begin{enumerate}[label=(\roman*)]}{\end{enumerate}}
\newenvironment{solution}{\par\noindent\P{} }{\ \qedsymbol}

\renewcommand{\vec}[1]{\ensuremath{\mathbf{#1}}}
\newcommand{\pd}[3][]{\left(\frac{\partial #2}{\partial #3}\right)_{#1}}




\begin{document}
%\maketitle

\setcounter{section}{\anumber}
\addtocounter{section}{-1}
\section{ --- PHY 494: Homework assignment (15 points total)}

\noindent Due Thursday, March 29, 2018, 11:59pm.

\noindent
%  \url{}
\fbox{\parbox{\linewidth}{Submission is to your \textbf{private
      GitHub repository}.}}
Read the following instructions carefully. Ask if anything is unclear.
\begin{enumerate}
\item \texttt{cd} into your assignment repository (change
  \emph{YourGitHubUsername} to your GitHub username) and run the
  update script \texttt{./scripts/update.sh} (replace
  \emph{YourGitHubUsername} with your GitHub username):
  \begin{minted}{bash}
    cd  assignments-2018-YourGitHubUsername
    bash ./scripts/update.sh
  \end{minted}
  It should create three subdirectories\footnote{If the script fails,
    file an issue in the
    \href{https://github.com/ASU-CompMethodsPhysics-PHY494/PHY494-assignments-skeleton/issues}{Issue
      Tracker for PHY494-assignments-skeleton} and just create the
    directories manually.} \texttt{assignment\_\anum/Submission},
  \texttt{assignment\_\anum/Grade}, and
  \texttt{assignment\_\anum/Work}.
\item You can try out code in the \texttt{assignment\_\anum/Work}
  directory but you don't have to use it if you don't want to. Your
  grade with comments will appear in
  \texttt{assignment\_\anum/Grade}.
\item Create your solution in
  \texttt{assignment\_\anum/Submission}. Use Git to \texttt{git
    add} files and \texttt{git commit} changes.

  You can create a PDF, a text file or Jupyter notebook inside the
  \texttt{assignment\_\anum/Submission} directory as well as Python
  code (if required). \textbf{Name your files \texttt{hw\anum.pdf} or
    \texttt{hw\anum.txt} or \texttt{hw\anum.ipynb}}, depending on how
  you format your work. Files with code (if requested) should be named
  exactly as required in the assignment.
\item When you are ready to submit your solution, do a final
  \texttt{git status} to check that you haven't forgotten anything,
  commit any uncommited changes, and \texttt{git push} to your GitHub
  repository. Check on \emph{your} GitHub repository web
  page\footnote{\texttt{https://github.com/ASU-CompMethodsPhysics-PHY494/assignments-2018-\emph{YourGitHubUsername}}}
  that your files were properly submitted.

  You can push more updates up until the deadline. Changes after the
  deadline will not be taken into account for grading.
\end{enumerate}
Homeworks must be legible and intelligible or may otherwise be
returned ungraded with 0 points.

\subsection{Square-root with Newton's method (15 points)}
\label{sec:sqrt}

The square root function\footnote{The symbol $\sqrt{\cdot}$ is
  commonly used to denote the operation of $q$, so that
  $q(x) \equiv \sqrt{x}$. For the following it is more useful to think
  of the square root as a special function than just a calculation.}
$q(x)$ can be \emph{defined} by the equation
\begin{gather}
  \label{eq:sqrtdef}
  q(x)^{2} = x.
\end{gather}
The goal is to develop and to implement an \textbf{efficient algorithm
to compute square roots}.

The defining equation Eq.~\ref{eq:sqrtdef} can be rearranged as 
\begin{align}
  q(x)^{2} - x &= 0   \label{eq:sqrtroot}\\
  q^{2} - x &= 0 \label{eq:q}\\
  f_{x}(q) &= 0 \label{eq:fqx}
\end{align}
where $f_{x}(q) = q^{2} - x$ is now considered a function of the
\emph{variable} $q$ and a given \emph{parameter} $x$. Finding the
square root $q(x)$ amounts to finding the root of $f_{x}(q)$, i.e.,
find that value $q$ that makes Eq.~\ref{eq:q} true for a given $x$.

\begin{enuma}
\item \label{li:NR} Use the iterative Newton-Raphson algorithm from the class to
  implement a function \texttt{sqrt(x, tol=1e-6, Nmax=100)} in a
  module \texttt{functions.py} that returns the square root of $x$ to
  a tolerance of \texttt{tol} and uses at a maximum \texttt{Nmax}
  iterations. If \texttt{Nmax} is exceeded print a warning message and
  return \texttt{None}.\footnote{In your code you may \emph{not} use a
    library square root function such as \texttt{math.sqrt} or
    \texttt{numpy.sqrt} nor taking fractional powers such as
    \texttt{x**0.5} or \texttt{numpy.power(x, 0.5)}.} Your code should
  guess a good starting value for the Newton-Raphson scheme,
  e.g,. $x/2$. 

  In the Newton-Raphson scheme you have to calculate the update
  $\Delta q$ to $q$ in order to obtain the new best guess for the root
  \begin{gather}
    \label{eq:update}
    q \leftarrow q + \Delta q
  \end{gather}
  with
  \begin{gather}
    \label{eq:NR}
    \Delta q = -\frac{f_{x}(q)}{f_{x}'(q)}
  \end{gather}
  In your code you may either use a finite difference scheme to
  calculate $f_{x}'(q)$ \emph{or} (more efficiently), use the
  \emph{analytical derivative} $f_{x}'(q) = 2q$
  directly.\footnote{Using the analytical derivatives makes for a
    handy algorithm to \emph{manually} calculate square roots and
    indeed this is how Newton came up with the method. The algorithm
    for computing the square root was also already known to the
    ancient Babylonians.}

  Your code must produce correct results, as tested with
  \texttt{test\_functions.py}.\footnote{Some specific tests are
    allowed to fail and they are marked with \texttt{x} or
    \texttt{xfail} in the test output --- this is ok.} You can run
  these tests yourself with
\begin{minted}{bash}
pytest -v test_functions.py
\end{minted}
  (in the same directory as your \texttt{functions.py}). \points{10}
\item \label{li:results} Show results for
  $x = 0, 0.456\times 10^{-8}, 10^{-3}, 0.1, 0.64, 0.99, 1, 5, 9, 12.5,
  10^{3}, 1.2345 \times 10^{8}$
  for a tolerance of $10^{-6}$. Put the results in a text file
  \texttt{sqrt.txt}. The results should be arranged so that each line
  contains $x$ and $q(x)$. \points{5}
% \item \label{li:analytical}\BONUS Do one Newton-Raphson step analytically and come up with
%   a handy way to calculate approximations to square roots. \bonus{3}
\end{enuma}

\noindent
Note: You don't have to submit any notebooks or written
text for problems \ref{li:NR} and \ref{li:results}. It will be sufficient to submit
\begin{itemize}
\item \texttt{functions.py}
\item \texttt{sqrt.txt}
\end{itemize}
% If you are also solving the \emph{Bonus problem \ref{li:analytical}}
% then include a text document in PDF format to show your solution.





\end{document}

%%% Local Variables: 
%%% mode: latex
%%% TeX-master: t
%%% End: 
