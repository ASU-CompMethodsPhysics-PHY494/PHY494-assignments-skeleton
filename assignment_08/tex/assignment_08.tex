%%% generic article type (pdf)latex file
%%% use together with Makefile

\documentclass[letterpaper]{scrartcl}
\usepackage{graphicx}
\usepackage{amsmath,amsfonts,amsthm,amsbsy}
\usepackage{eufrak}
\usepackage{mathabx}
\usepackage{courier}
\usepackage{url}
\usepackage{color}
\usepackage[usenames,dvipsnames,svgnames,table]{xcolor}
\usepackage{hyperref}
\hypersetup{
     colorlinks   = true,
     urlcolor     = blue,
     linkcolor    = red,
     citecolor    = black
}


\usepackage{enumitem}
\usepackage{booktabs}
\usepackage{cprotect}
\usepackage{minted}

%\usepackage{wrapfig}
%\usepackage{subfig}
%\usepackage[format=plain,labelsep=period,font=small,labelfont=bf]{caption}

%------------------------------------------------------------
% assignment
%
\newcommand{\anumber}{8}
%
%------------------------------------------------------------
\newcommand{\anum}{0\anumber}

% hyperref https://en.wikibooks.org/wiki/LaTeX/Hyperlinks#.5Chref
\urlstyle{same}


%% not working yet...
\newcounter{TotalPoints}
\newcounter{TotalBonus}

\newcommand{\BONUS}{\textsc{Bonus: }}
\newcommand{\bonus}[1]{\textbf{[bonus +#1*]}\stepcounter{TotalBonus}}
\newcommand{\points}[1]{\textbf{[#1 points]}\stepcounter{TotalPoints}}
\newenvironment{enuma}{\begin{enumerate}[label=(\alph*)]}{\end{enumerate}}
\newenvironment{enumi}{\begin{enumerate}[label=(\roman*)]}{\end{enumerate}}
\newenvironment{solution}{\par\noindent\P{} }{\ \qedsymbol}

\renewcommand{\vec}[1]{\ensuremath{\mathbf{#1}}}
\newcommand{\pd}[3][]{\left(\frac{\partial #2}{\partial #3}\right)_{#1}}




\begin{document}
%\maketitle

\setcounter{section}{\anumber}
\addtocounter{section}{-1}
\section{ --- PHY 494: Homework assignment (40 points total)}

\noindent Due \textbf{Tuesday} (after Spring Break), March 17, 2020, 11:59pm.

\noindent
%  \url{}
\fbox{\parbox{\linewidth}{Submission is to your \textbf{private
      GitHub repository}.}}
Read the following instructions carefully. Ask if anything is unclear.
\begin{enumerate}
\item \texttt{cd} into your assignment repository (change
  \emph{YourGitHubUsername} to your GitHub username) and run the
  update script \texttt{./scripts/update.sh} (replace
  \emph{YourGitHubUsername} with your GitHub username):
  \begin{minted}{bash}
    cd  assignments-2020-YourGitHubUsername
    bash ./scripts/update.sh
  \end{minted}
  It should create three subdirectories\footnote{If the script fails,
    file an issue in the
    \href{https://github.com/ASU-CompMethodsPhysics-PHY494/PHY494-assignments-skeleton/issues}{Issue
      Tracker for PHY494-assignments-skeleton} and just create the
    directories manually.} \texttt{assignment\_\anum/Submission},
  \texttt{assignment\_\anum/Grade}, and
  \texttt{assignment\_\anum/Work}.
\item You can try out code in the \texttt{assignment\_\anum/Work}
  directory but you don't have to use it if you don't want to. Your
  grade with comments will appear in
  \texttt{assignment\_\anum/Grade}.
\item Create your solution in
  \texttt{assignment\_\anum/Submission}. Use Git to \texttt{git
    add} files and \texttt{git commit} changes.

  You can create a PDF, a text file or Jupyter notebook inside the
  \texttt{assignment\_\anum/Submission} directory as well as Python
  code (if required). \textbf{Name your files \texttt{hw\anum.pdf} or
    \texttt{hw\anum.txt} or \texttt{hw\anum.ipynb}}, depending on how
  you format your work. Files with code (if requested) should be named
  exactly as required in the assignment.
\item When you are ready to submit your solution, do a final
  \texttt{git status} to check that you haven't forgotten anything,
  commit any uncommited changes, and \texttt{git push} to your GitHub
  repository. Check on \emph{your} GitHub repository web
  page\footnote{\texttt{https://github.com/ASU-CompMethodsPhysics-PHY494/assignments-2020-\emph{YourGitHubUsername}}}
  that your files were properly submitted.

  You can push more updates up until the deadline. Changes after the
  deadline will not be taken into account for grading.
\end{enumerate}
Homeworks must be legible and intelligible or may otherwise be
returned ungraded with 0 points.

If you implement the function as specified you can run the tests in
the file \texttt{Submission/test\_hw08.py} with \texttt{pytest}
\begin{minted}{bash}
  cd Submission
  pytest -v test_outerplanets.py
\end{minted}
and all tests should pass. If you have errors, have a look at the
output and try to figure out what is still not working. Having the
tests pass is not a guarantee that you will get full points (but it is
general a very good sign!).

\textbf{Collaboration:} Up to three students may form a team and solve
the homework together. Each student may submit the same solution to
their own repo. \textbf{Add a text file \texttt{COLLABORATION.txt}} to
the repository in which you
\begin{itemize}
\item list the names of all team members (last name, first name), and
\item provide a brief statement ($<$ half a page) as to who in the
  team contributed what to the solution
\end{itemize}
If you solved the problem on your own, just put your own name as the
single name in \texttt{COLLABORATION.txt} and state in the file ``I
solved and completed the homework by myself.''.

\subsection{Discovery of Neptune (3-body problem) (40 points)}
\label{sec:neptune}
Historically, the planet Neptune was discovered because of its (small)
observable influence on the orbit of Uranus. Neptune attracts Uranus
and this leads to small changes in Uranus' orbital velocity. In this
problem you are studying a simplified situation only containing the
Sun, Uranus and Neptune.\footnote{For more background, see e.g.,
  \emph{Computational Physics} 9.7.}

Assume that to a first approximation that the orbits of Uranus and
Neptune are circular (so that you can easily calculate the initial
angular velocity)\footnote{In your simulation, do not enforce circular
orbits, though. With the given parameters, orbits should turn out to
be nearly circular.} and co-planar. The initial position is given by
an angle $\theta$ relative to the $x$-axis and the distance $r$ from
the sun, as listed in Table~\ref{tab:parameters}.

\begin{table}[bt]
  \centering
  \begin{tabular}{lcccc}
    \toprule
    planet & mass ($M_{\text{sol}}$) & distance (AU) 
           & orbital period (yr) & angular position (1690) $\theta$\\
    \midrule
    Sun    &      1 & --- & --- & ---\\
    Uranus & $4.366244\times 10^{-5}$ & 19.1914 & 84.0110 &
                                                            205.640$^{\circ}$\\
    Neptune & $5.151389\times 10^{-5}$ & 30.0611 & 164.7901 & 
                                                              288.380$^{\circ}$\\
    \bottomrule
  \end{tabular}
  \caption{Parameters for Uranus and Neptune}
  \label{tab:parameters}
\end{table}

Calculate the variation in the angular velocity of Uranus
$\Delta\omega_U(t)$ due to the influence of Neptune: Compare the
instantaneous angular velocity\footnote{The angular velocity $\omega$
  for circular motion is
  \begin{gather*}
    v = \frac{2\pi r}{T} = \omega r\\
    \omega = \frac{2\pi}{T} = \frac{v}{r}
  \end{gather*}
  where $T$ is the period and $v$ the speed (velocity is normal to the
  radial vector $\vec{r}$, $r$ is the distance from the center).}
$\omega_U(t)$ to the one in the absence of Neptune, $\omega^0_U$:
\begin{gather}
  \Delta\omega_U(t) = \omega_U(t) - \omega^0_U\label{eq:DeltaOmega}
\end{gather}
by plotting $\Delta\omega_U(t)$ over $t$ and briefly comment on your
result.

\subsubsection{General considerations and requirements}
\label{sec:general}

\begin{itemize}
\item Use astronomical units: AU for length (1 AU is the distance of
  the earth from the sun), year for time, mass in terms of solar
  masses (i.e.\ sun's mass $M = 1$) --- all as provided in
  Table~\ref{tab:parameters}, gravitational constant (in
  Eq.~\ref{eq:gravitation}) in AU: $G = 4\pi^{2}$
\item Use the \emph{Velocity Verlet} algorithm\footnote{Hidden \BONUS
    Try using another integration algorithm such as RK4 and compare
    the results to the calculation with Velocity Verlet. \bonus{5}}
  with a time step of 0.1 years to integrate the equations of motions
  of Uranus and Neptune.
\item The sun can be considered stationary ($M \gg m_U \approx m_N$).
\item Analyze 160 years (almost one complete orbit of Neptune).
\item Skeleton code is provided in \texttt{outerplanets.py}; you can
  use it but you don't have to. However, you must write your code so
  that
  \begin{enumerate}
  \item it can be imported as \texttt{import outerplanets}
  \item it contains a function \texttt{integrate\_orbits(dt=0.1,
      t\_max=320, coupled=True)} to do the integration; the function
    has to return a tuple \texttt{time, r, v} (see
    \texttt{outerplanets.py} for details).
  \end{enumerate}
  A test case \texttt{test\_outerplanets.py} is provided; you can run
  \begin{minted}{bash}
    pytest -v test_outerplanets.py
  \end{minted}
  and all tests should pass if you have done everything correctly.
\item Also plot the orbits of Uranus and Neptune.
\item The file \texttt{outerplanets.py} must be a separate file.
  Provide the two figures as individual PDFs\footnote{If PDF does not
    work for whatever reason, use the JPG format instead of PDF.} (use
  \texttt{ax.figure.savefig("filename.pdf")}). Other descriptions and
  code can be submitted as a notebook or in other suitable forms.
\end{itemize}

To calculate $\Delta\omega$ in Eq.~\ref{eq:DeltaOmega} you need to run
your integration 
\begin{enumerate}
\item \emph{without} the interaction between Neptune and Uranus
  (\texttt{coupled=False}) and 
\item \emph{with} the gravitational interaction.
\end{enumerate}

\subsubsection{Angular velocity}
\label{sec:velocity}
You can calculate the instantaneous angular velocity as\footnote{The
  code contains the function \texttt{omega()} to perform the
  calculation.}
\begin{gather*}
  \omega(t) = \frac{v(t)}{r(t)}, \quad r=|\vec{r}|,\ v=|\vec{v}|
\end{gather*}
where position $\vec{r}(t)$ and velocity $\vec{v}(t)$ are determined
from the integrator.

\subsubsection{Interactions}
\label{sec:F}

This is a \emph{three-body} problem. You must compute the forces for
Uranus and for Neptune, using the three interactions 
\begin{enumerate}
\item Uranus --- Sun, $\vec{F}_{U,S}$
\item Neptune --- Sun, $\vec{F}_{N,S}$
\item Uranus -- Neptune, $\vec{F}_{U,N}$
\end{enumerate}
Thus, for example, the force on Uranus is
$\vec{F}_{U} = \vec{F}_{U,S} + \vec{F}_{U,N}$.
%
Newton's force law for gravitation is
\begin{align}
\label{eq:gravitation}
\vec{F} &= -\frac{G m M}{r^2} \hat{\vec{r}}\\
\hat{\vec{r}} &= \frac{1}{\sqrt{x^2 + y^2}} \left(\begin{array}{c} x
                                                    \\
                                                    y \end{array}\right) \notag
\end{align}
where $G$ is the gravitational constant, $m$ and $M$ are the masses of
the two bodies, and $\vec{r}$ is the vector between the positions of
the two bodies. Remember that Newton's third law states
\begin{gather}
  \label{eq:NewtonTwo}
  \vec{F}_{12} = -\vec{F}_{21}.
\end{gather}


%Total Points: \arabic{TotalPoints}. Total Bonus: \arabic{TotalBonus}*


\end{document}

%%% Local Variables: 
%%% mode: latex
%%% TeX-master: t
%%% End: 
