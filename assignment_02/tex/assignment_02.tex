%%% generic article type (pdf)latex file
%%% use together with Makefile

\documentclass[letterpaper]{scrartcl}
\usepackage{graphicx}
\usepackage{amsmath,amsfonts,amsthm}
\usepackage{eufrak}
\usepackage{mathabx}
\usepackage{url}
\usepackage[usenames,dvipsnames,svgnames,table]{xcolor}
\usepackage[colorlinks]{hyperref}
\hypersetup{
     colorlinks   = true,
     urlcolor     = blue,
     linkcolor    = red,
     citecolor    = black
}
\usepackage{enumitem}
\usepackage{booktabs}
\usepackage{cprotect}
\usepackage{minted}

% https://tex.stackexchange.com/a/340588
\usepackage{tikz}
\usepackage[edges]{forest}

%\usepackage{wrapfig}
%\usepackage{subfig}
\usepackage[format=plain,labelsep=period,font=small,labelfont=bf]{caption}

%------------------------------------------------------------
% assignment
%
\newcommand{\anumber}{2}
%
%------------------------------------------------------------

% hyperref https://en.wikibooks.org/wiki/LaTeX/Hyperlinks#.5Chref
\urlstyle{same}

%% not working yet...
\newcounter{TotalPoints}
\newcounter{TotalBonus}

\newcommand{\BONUS}{\textsc{Bonus: }}
\newcommand{\bonus}[1]{\textbf{[bonus +#1*]}\stepcounter{TotalBonus}}
\newcommand{\points}[1]{\textbf{[#1 points]}\stepcounter{TotalPoints}}
\newenvironment{enuma}{\begin{enumerate}[label=(\alph*)]}{\end{enumerate}}
\newenvironment{enumi}{\begin{enumerate}[label=(\roman*)]}{\end{enumerate}}
\newenvironment{solution}{\par\noindent\P{} }{\ \qedsymbol}

\renewcommand{\vec}[1]{\ensuremath{\mathbf{#1}}}
\newcommand{\pd}[3][]{\left(\frac{\partial #2}{\partial #3}\right)_{#1}}

\newcommand{\anum}{0\anumber}


% dir trees from Guilherme Zanotelli
% https://tex.stackexchange.com/a/340588
\definecolor{foldercolor}{RGB}{124,166,198}

\tikzset{pics/folder/.style={code={%
    \node[inner sep=0pt, minimum size=#1](-foldericon){};
    \node[folder style, inner sep=0pt, minimum width=0.3*#1, minimum height=0.6*#1, above right, xshift=0.05*#1] at (-foldericon.west){};
    \node[folder style, inner sep=0pt, minimum size=#1] at (-foldericon.center){};}
    },
    pics/folder/.default={20pt},
    folder style/.style={draw=foldercolor!80!black,top color=foldercolor!40,bottom color=foldercolor}
}

\forestset{is file/.style={edge path'/.expanded={%
        ([xshift=\forestregister{folder indent}]!u.parent anchor) |- (.child anchor)},
        inner sep=1pt},
    this folder size/.style={edge path'/.expanded={%
        ([xshift=\forestregister{folder indent}]!u.parent anchor) |- (.child anchor) pic[solid]{folder=#1}}, inner xsep=0.6*#1},
    folder tree indent/.style={before computing xy={l=#1}},
    folder icons/.style={folder, this folder size=#1, folder tree indent=3*#1},
    folder icons/.default={12pt},
}


\begin{document}
%\maketitle

\setcounter{section}{\anumber}
\addtocounter{section}{-1}
\section{ --- PHY 494: Homework assignment (51 points total)}

\noindent Due Thursday, January 30, 2020, 11:59pm.

\noindent Submit a PDF through Canvas (name it
\texttt{\emph{lastname}\_\emph{firstname}\_hw\anumber.pdf}).\footnote{This
is the last submission through Canvas. All future assignments will be
submitted to your private git repository.}
Homeworks must be legible or may otherwise be returned ungraded with 0
points.


This assignment contains \textbf{bonus problems}. A bonus problem is
optional. If you do it you get additional points that count towards
this homework's total, although you can't get more than the maximum
number of points. If you don't do it you can still get full
points. Bonus problems and bonus points are indicated with an asterisk
``*''.

Note: In general, for full credit you need to (1) show the commands
that you used and (2) answer the question. Sometimes you should also
copy and paste output.

\subsection{Python operators (6 points)}

\begin{enuma}
\item What does the following code output? What are the names of the four
  mathematical operations that are carried out in lines 4--7? \points{4}
  \begin{minted}[linenos]{python}
    a = 5.0
    b = 10
    
    result_1 = a + b
    result_2 = a / b
    result_3 = a ** b
    result_4 = result_3 % result_1
    
    print(result_1, result_2, result_3, result_4)
  \end{minted}
\item What error does the following code produce and why? \points{2}
  \begin{minted}{python}
  x = 1.234
  y = -0.5

  result = (x*y - (-y/x)**0.5)/(1 + (y//x)**3)
  \end{minted}
\end{enuma}

\subsection{Very Simple Temperature Calculator (13 points)}
\label{sec:calculator}

Write a Python program \texttt{addtemperatures.py} that adds a
temperature difference in Fahrenheit, $\Delta\theta$, to an absolute
temperature, given in Kelvin, $T$, and returns the sum in Kelvin. The
program should
\begin{itemize}
\item ask the user for two floating point numbers $T$ (absolute
  temperature in Kelvin) and $\Delta\theta$ (temperature difference in
  degrees Fahrenheit) as input
\item store the sum ``$T + \Delta\theta$'' in units of Kelvin (where
  $\Delta\theta$ must be converted to Kelvin) in a variable named
  \texttt{T\_total} and also print it.
\end{itemize}
The conversion of an absolute temperature from Fahrenheit to Kelvin is
(written with numbers $T/\text{K}$ and
$\theta/^{\circ}\text{F}$)\footnote{We use ``symbol/unit'' to indicate
a number without the unit so that we can write equations where all
units correctly balance. For instance, if $T=373$~K then $T/\text{K}$
is the number 373. This approach is more precise than just saying
``Take $T$ as the temperature in Kelvin and $\theta$ in Fahrenheit.''}
\begin{gather}
  \label{eq:conversion}
  T/\text{K} = \frac{5}{9}(\theta/\text{$^{\circ}$F} - 32) + 273.15
\end{gather}

\begin{enuma}
\item Derive an expression for $\Delta T = T_{2} - T_{1}$ as a
  function of $\Delta \theta = \theta_{2} - \theta_{1}$ (where
  $T_{2}$, $T_{1}$, $\theta_{2}$, and $\theta_{1}$ are arbitrary and
  only introduced to make the connection to
  Eq.~\ref{eq:conversion}). Show that the \emph{difference in Kelvin}
  is 5/9-th of the difference in Fahrenheit, \points{3}
  \begin{gather}
    \label{eq:diff}
    \Delta T/\text{K} = \frac{5}{9} \Delta \theta/\text{$^{\circ}$F}.
  \end{gather}
\item Use your result from the previous problem (i.e.,
  Eq.~\ref{eq:diff}) to derive a mathematical expression to compute
  the sum of $T$ and $\Delta\theta$. \points{2}  
\item Write the \texttt{addtemperatures.py} program and copy and paste
  the code. \points{7}
\item Show the complete input and output (copy and paste) for the
  input $\Delta\theta = 63^{\circ}\text{F}$ and $T =
  265\,\text{K}$. \points{4}
\end{enuma}


\subsection{Version control with Git (9 points)}
Keep your answers short, one or two sentences should be sufficient for
questions \ref{li:gitinit}--\ref{li:gitpush}.
\begin{enuma}
\item \emph{Briefly} explain what a version control system such as Git
  does and how it can help you. (For your answer, it is sufficient to
  focus on three different aspects out of many --- choose the ones
  that \emph{you} find most important.) \points{3}
\item \label{li:gitinit}Explain the difference between \texttt{git init} and \texttt{git
  clone}. \points{2}
\item Explain the difference between \texttt{git add} and \texttt{git
    commit}. \points{2}
\item \label{li:gitpush} Explain the difference between \texttt{git commit} and
  \texttt{git push} \points{2}
\end{enuma}

\subsection{Be the Git (10 points)}

In this problem you should state various outcomes when a number of git
commands are run on the  directory structure in Figure~\ref{fig:tree}
(\emph{Documents} is the top-level directory, sub-directories are
indicated by blue folder icons, files are text):\footnote{You can
  pretend to be git or you can also create the directory structure
  yourself and run the commands.}

\begin{figure}[!h]
  \centering
  \begin{forest}
    for tree={font=\sffamily, grow'=0, folder indent=.9em, folder
      icons, edge=densely dotted} [Documents [planets [hoth.jpg, is
    file] [tatooine.jpg, is file] [alderaan.jpg, is file] ] [vehicles
    [ships [destroyer.txt, is file] [interceptor.txt, is file]]
    [stations [deathstar.xxx, is file]]] [jedi\_locations.map, is
    file] [shopping.txt, is file] [TODO, is file] ]
  \end{forest}  
  \caption{Part of the directory tree of user dvader. \emph{Documents}
    is the top-level directory, sub-directories are indicated by blue
    folder icons, files are just shown as text with their file name,
    e.g., \emph{TODO} or \emph{hoth.jpg}.}
  \label{fig:tree}
\end{figure}

User dvader decides (quite sensibly) to use version control, namely
\textbf{git}, for his Documents. The \emph{Documents} directory was
not under version control before.  The first set of commands that he
performs is:
\begin{minted}{bash}
cd Documents
git init
git add shopping.txt
git add planets/alderaan.jpg
\end{minted}

To answer the following question, create a table where you list the
command that is being executed, the \emph{files in the git staging
  area} and the \emph{files in the git repository}. After the initial
commands, your table should look like this:

\begin{center}
  \small
  \begin{tabular}{lp{0.3\linewidth}p{0.15\linewidth}}
    \toprule
    command & staging area & repository\\
    \midrule
    git init & & \\
    git add shopping.txt & shopping.txt & \\
    git add planets/alderaan.jpg & shopping.txt, planets/alderaan.jpg & \\
    \bottomrule                           
  \end{tabular}
\end{center}

List \emph{all} files that are in either staging area or repository,
not just new ones. If there are no files (as in the repository during
the first three \texttt{git} commands), leave the space empty.

Add the table above to your submitted solution and continue it by
listing all the files in the staging area and in the repository, after
each of the following commands have been carried out:
\begin{minted}{bash}
git commit
git add vehicles/stations
git commit
git add vehicles TODO
git commit
git rm planets/alderaan.jpg
git add planets
git commit
\end{minted}



\subsection{Your GitHub account (10 points)}

As part of the last lesson you should have
\href{http://asu-compmethodsphysics-phy494.github.io/ASU-PHY494/page2/#set-up-your-own-github-repositories}{set
  up your own GitHub account} on \url{https://github.com} (if you have
not done it yet, do it now!). What is your \textbf{GitHub username}?
\begin{itemize}
\item Write down your GitHub username. \points{10}
\item Take the survey
  \href{https://forms.gle/gt8PYQ931eU1ULFk9}{PHY
    494: Your GitHub account}\footnote{In case the link to the survey
    is not clickable: got to
    \url{https://forms.gle/gt8PYQ931eU1ULFk9}. You must be logged
    in with your ASU account. Log in to \url{https://my.asu.edu} first
    and then go to the survey.} if you have not done so already.
\end{itemize}




%Total Points: \arabic{TotalPoints}. Total Bonus: \arabic{TotalBonus}*


\end{document}

%%% Local Variables: 
%%% mode: latex
%%% TeX-master: t
%%% End: 
