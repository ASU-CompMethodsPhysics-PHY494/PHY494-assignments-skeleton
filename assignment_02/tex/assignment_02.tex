%%% generic article type (pdf)latex file
%%% use together with Makefile

\documentclass[letterpaper]{scrartcl}
\usepackage{graphicx}
\usepackage{amsmath,amsfonts,amsthm}
\usepackage{eufrak}
\usepackage{mathabx}
\usepackage{url}
\usepackage[colorlinks]{hyperref}
\usepackage{enumitem}
\usepackage{booktabs}
\usepackage{minted}


%\usepackage{wrapfig}
\usepackage{subfig}
\usepackage[format=plain,labelsep=period,font=small,labelfont=bf]{caption}

%------------------------------------------------------------
% assignment
%
\newcommand{\anumber}{2}
%
%------------------------------------------------------------

\newcommand{\BONUS}{\textsc{Bonus: }}
\newcommand{\bonus}[1]{\textbf{[bonus +#1*]}}
\newcommand{\points}[1]{\textbf{[#1 points]}}
\newenvironment{enuma}{\begin{enumerate}[label=(\alph*)]}{\end{enumerate}}
\newenvironment{enumi}{\begin{enumerate}[label=(\roman*)]}{\end{enumerate}}
\newenvironment{solution}{\par\noindent\P{} }{\ \qedsymbol}

\renewcommand{\vec}[1]{\ensuremath{\mathbf{#1}}}
\newcommand{\pd}[3][]{\left(\frac{\partial #2}{\partial #3}\right)_{#1}}

\newcommand{\fnhref}[2]{\href{#1}{#2}\footnote{\url{#1}}}

\newcommand{\python}[1]{\mintinline{python}{#1}}

\begin{document}
%\maketitle

\setcounter{section}{\anumber}
\addtocounter{section}{-1}
\section{ --- PHY 494: Homework assignment (62 points total)}

\noindent Due Sunday, Jan 28, 2018, 11:59pm.

\noindent Submit a PDF through Blackboard (name it
\texttt{\emph{lastname}\_\emph{firstname}\_hw\anumber.pdf}).
Homeworks must be legible or may otherwise be returned ungraded with 0
points.

This assignment contains \textbf{bonus problems}. A bonus problem is
optional. If you do it you get additional points that count towards
this homework's total, although you can't get more than the maximum
number of points. If you don't do it you can still get full
points. Bonus problems and bonus points are indicated with an asterisk
``*''.

Note: In general, for full credit you need to (1) show the commands
that you used and (2) answer the question. Sometimes you should also
copy and paste output.

\subsection{Python data types (7 points)}
What is the Python data type of each of the following values?
\begin{enuma}
\item \python{3.14515} \points{1}
\item \python{0} \points{1}
\item \python{False} \points{1}
\item \python{'To be or not to be'} \points{1}
\item \python{[3, 2, 1, "lift off!"]} \points{1}
\item \python{3, 2, 1, "lift off!"} \points{1}
\item \python{None} \points{1}
\item \BONUS \python{{'name': 'Hamlet', 'occupation': 'Prince'}} \bonus{1}
\end{enuma}

\subsection{Operators (8 points)}

\begin{enuma}
\item What does the following code output? What are the names of the three
  mathematical operations that are carried out in lines 4--6? \points{4}
  \begin{minted}[linenos]{python}
    a = 5.0
    b = 10
    
    result_1 = a + b
    result_2 = a / b
    result_3 = a ** b
    
    print(result_1, result_2, result_3)
  \end{minted}
\item What error does the following code produce and why? \points{4}
  \begin{minted}{python}
    name = 'Hamlet, Prince of Denmark'
    age = 30
    print(name + age)
  \end{minted}
  % http://www.shakespeare-online.com/plays/hamlet/hamletsage.html
\end{enuma}


\subsection{Python Lists and Strings (20 points)}

Lists and strings share some similarities but also have important
differences. Let's look at them. (Type code in the Python interpreter
e.g., \texttt{ipython}).

\begin{minted}{python}
bag = ["guide", "towel", "tea", 42]
ga = "Four score and seven years ago"
\end{minted}

(Note that spaces are shown explicitly in the second string with the
symbol ``\textvisiblespace''---just type a space.)

\begin{enuma}
\item How do you have to slice \python{bag} in order to get
  \python{['towel', 'tea']}? \points{1}
\item What does \python{bag[::-1]} do?  
  
  How do you slice bag in order to get
  \python{['tea', 'towel']}?  \points{2}
\item Strings can also be sliced. How do you have to slice \python{ga}
  to get
  \begin{itemize}
  \item   "Four"
  \item   "seven"
  \end{itemize}
  \points{2}
\item You can access elements of a list in a variety of ways: 
  \begin{enumi}
  \item   Explain what
\begin{minted}{python}
bag[0] = 'book'
\end{minted}
    does? (Hint: print \python{bag}!) \points{1}
  \item  Create two new variables:
\begin{minted}{python}
mybag = bag
yourbag = bag[:]
\end{minted}
    and use them:
\begin{minted}{python}
mybag[3] = "mice" 
yourbag.append("money")
\end{minted}
    What is the content of \python{bag}, \python{mybag},
    \python{yourbag}? \points{2}
  \item  From your observations, explain how the assignment \python{x = a} differs
    from \python{y = a[:]}? \points{3}
  \end{enumi}
\item  Try
\begin{minted}{python}
  ga[:4] = "Three"
\end{minted}
  \begin{enumi}
  \item Describe what happens?\footnote{Note that strings are
      ``immutable'' objects in Python whereas lists are ``mutable''.}
    \points{1}
  \item How would you construct the string \python{"Three score and
      seven years ago"} from \python{ga} and the string
    \python{"Three"}? \points{1}
  \end{enumi}
\item What do the commands
\begin{minted}{python}
ga.split()
a, b, c = ga.split()[:3]
list([1,2,3])
list(ga)
\end{minted}
  do? You can show the output but you need to explain in your own
  words what is happening. \points{4}
\item Nested lists: Given the list
\begin{minted}{python}
bags = [['salt', 'pepper'], ['pen', 'eraser', 'ruler']]
\end{minted}
  how do you have to index \python{bags} to get

  \begin{enumi}
  \item \python{['salt', 'pepper']} \points{1}
  \item \python{'pepper'} \points{1}
  \item \python{'ruler'} \points{1}
  \end{enumi}
\end{enuma}

\subsection{Very Simple Temperature Calculator (13 points)}
\label{sec:calculator}

Write a Python program \texttt{addtemperatures.py} that adds a
temperature difference in Fahrenheit, $\Delta\theta$, to an absolute
temperature, given in Kelvin, $T$. The program should
\begin{itemize}
\item ask the user for two floating point numbers $T$ (absolute
  temperature in Kelvin) and $\Delta\theta$ (temperature difference in
  degrees Fahrenheit) as input
\item print the sum ``$T + \Delta\theta$'' in units of Kelvin (where
  $\Delta\theta$ must be converted to Kelvin)
\end{itemize}
The conversion of an absolute temperature from Fahrenheit to Kelvin is
(written with numbers $T/\text{K}$ and
$\theta/^{\circ}\text{F}$)\footnote{We use ``symbol/unit'' to indicate
a number without the unit so that we can write equations where all
units correctly balance. For instance, if $T=373$~K then $T/\text{K}$
is the number 373. This approach is more precise than just saying
``Take $T$ as the temperature in Kelvin and $\theta$ in Fahrenheit.''}
\begin{gather}
  \label{eq:conversion}
  T/\text{K} = \frac{5}{9}(\theta/\text{$^{\circ}$F} - 32) + 273.15
\end{gather}

\begin{enuma}
\item \BONUS Derive an expression for $\Delta T = T_{2} - T_{1}$ as a
  function of $\Delta \theta = \theta_{2} - \theta_{1}$ (where
  $T_{2}$, $T_{1}$, $\theta_{2}$, and $\theta_{1}$ are arbitrary and
  only introduced to make the connection to
  Eq.~\ref{eq:conversion}). Show that the the difference in Kelvin is
  5/9-th of the difference in Fahrenheit, \bonus{3}
  \begin{gather}
    \label{eq:diff}
    \Delta T/\text{K} = \frac{5}{9} \Delta \theta/\text{$^{\circ}$F}.
  \end{gather}
\item Use your result from the previous problem (i.e.,
  Eq.~\ref{eq:diff}) to derive a mathematical expression to compute
  the sum of $T$ and $\Delta\theta$. \points{2}  
\item Write the \texttt{addtemperatures.py} program and copy and paste
  the code. \points{7}
\item Show the complete input and output (copy and paste) for the
  input $\Delta\theta = 63^{\circ}\text{F}$ and $T =
  265\,\text{K}$. \points{4}
\end{enuma}


\subsection{Loops (14 points)}
\begin{enuma}
\item Use a \python{for} loop to print each element of the list
  \begin{minted}{python}
    sentence = ["We", "must", "walk", "before", "we", "can", "run"]
  \end{minted}
  Show your code and your output. \points{4}
\item \BONUS Find a way to only print every other line from
  \python{sentence}, i.e., the output should read ``We walk we
  run''. Show your code and your output. \bonus{2}
\item Use a loop to sum the integers from 1 to 1000 and print the
  final result (500500). \points{4}

  \noindent\emph{Note}: You can increment a variable \python{total}
  with a value \python{x} with the assignment
  \begin{minted}{python}
    total = total + x
  \end{minted}
  or equivalently but more compactly written
  \begin{minted}{python}
    total += x
  \end{minted}
  Both expressions do the same thing: add two values together and then
  assign the sum to the variable \python{total}, overwriting the old
  value of \python{total} in the process.
\item Write a program that counts down from 10 to 0 and prints each
  number (``10 9 8 ... 3 2 1'').
  \begin{enumi}    
  \item Use a \python{while} loop. \points{3}
  \item Use a \python{for} loop. \points{3}
  \end{enumi}
  (You can use any other Python functions that you might need, such as
  \python{range}.)

  Show your code and output.
\end{enuma}



%%% Local Variables:
%%% mode: latex
%%% TeX-master: t
%%% End:



\end{document}

%%% Local Variables: 
%%% mode: latex
%%% TeX-master: t
%%% End: 
