%%% generic article type (pdf)latex file
%%% use together with Makefile

\documentclass[letterpaper]{scrartcl}
\usepackage{graphicx}
\usepackage{amsmath,amsfonts,amsthm}
\usepackage{eufrak}
\usepackage{mathabx}
\usepackage{url}
\usepackage[usenames,dvipsnames,svgnames,table]{xcolor}
\usepackage[colorlinks]{hyperref}
\hypersetup{
     colorlinks   = true,
     urlcolor     = blue,
     linkcolor    = red,
     citecolor    = black
}
\usepackage{enumitem}
\usepackage{booktabs}
\usepackage{cprotect}
\usepackage{minted}


%\usepackage{wrapfig}
%\usepackage{subfig}
%\usepackage[format=plain,labelsep=period,font=small,labelfont=bf]{caption}

%------------------------------------------------------------
% assignment
%
\newcommand{\anumber}{4}
%
%------------------------------------------------------------

% hyperref https://en.wikibooks.org/wiki/LaTeX/Hyperlinks#.5Chref
\urlstyle{same}

%% not working yet...
\newcounter{TotalPoints}
\newcounter{TotalBonus}

\newcommand{\BONUS}{\textsc{Bonus: }}
\newcommand{\bonus}[1]{\textbf{[bonus +#1*]}\stepcounter{TotalBonus}}
\newcommand{\points}[1]{\textbf{[#1 points]}\stepcounter{TotalPoints}}
\newenvironment{enuma}{\begin{enumerate}[label=(\alph*)]}{\end{enumerate}}
\newenvironment{enumi}{\begin{enumerate}[label=(\roman*)]}{\end{enumerate}}
\newenvironment{solution}{\par\noindent\P{} }{\ \qedsymbol}

\renewcommand{\vec}[1]{\ensuremath{\mathbf{#1}}}
\newcommand{\pd}[3][]{\left(\frac{\partial #2}{\partial #3}\right)_{#1}}

\newcommand{\anum}{0\anumber}




\begin{document}
%\maketitle

\setcounter{section}{\anumber}
\addtocounter{section}{-1}
\section{ --- PHY 494: Homework assignment (33 points total)}

\noindent Due Thursday, Feb 15, 2018, 11:59pm.

\noindent
\fbox{\parbox{\linewidth}{Submission is now to your \textbf{private
      GitHub repository}. Follow the link provided to you by the
    instructor in order for the repository to be set up: It will have
    the name
    \emph{ASU-CompMethodsPhysics-PHY494/assignments-2018-\emph{YourGitHubUsername}}
    and will only be visible to you and the instructor/TA. Follow the
    instructions below to submit this (and all future) homework.}}
Read the following instructions carefully. Ask if anything is unclear.
\begin{enumerate}
\item \texttt{git clone} your assignment repository (change
  \emph{YourGitHubUsername} to your GitHub username)
  \begin{minted}{bash}
    repo="assignments-2018-YourGitHubUsername.git" 
    git clone https://github.com/ASU-CompMethodsPhysics-PHY494/${repo}
  \end{minted}
\item run the script
  \texttt{./scripts/update.sh} (replace \emph{YourGitHubUsername} with
  your GitHub username):
  \begin{minted}{bash}
    cd ${repo} 
    bash ./scripts/update.sh
  \end{minted}
  It should create three subdirectories\footnote{If the script fails,
    file an issue in the
    \href{https://github.com/ASU-CompMethodsPhysics-PHY494/PHY494-assignments-skeleton/issues}{Issue
      Tracker for PHY494-assignments-skeleton} and just create the
    directories manually.} \texttt{assignment\_\anum/Submission},
  \texttt{assignment\_\anum/Grade}, and
  \texttt{assignment\_\anum/Work}.
\item You can try out code in the \texttt{assignment\_\anum/Work}
  directory but you don't have to use it if you don't want to. Your
  grade with comments will appear in
  \texttt{assignment\_\anum/Grade}.
\item Create your solution in
  \texttt{assignment\_\anum/Submission}. Use Git to \texttt{git
    add} files and \texttt{git commit} changes.

  You can create a PDF, a text file or Jupyter notebook inside the
  \texttt{assignment\_\anum/Submission} directory as well as Python
  code (if required). \textbf{Name your files \texttt{hw\anum.pdf} or
    \texttt{hw\anum.txt} or \texttt{hw\anum.ipynb}}, depending on how
  you format your work. Files with code (if requested) should be named
  exactly as required in the assignment.
\item When you are ready to submit your solution, do a final
  \texttt{git status} to check that you haven't forgotten anything,
  commit any uncommited changes, and \texttt{git push} to your GitHub
  repository. Check on \emph{your} GitHub repository web
  page\footnote{\texttt{https://github.com/ASU-CompMethodsPhysics-PHY494/assignments-2018-\emph{YourGitHubUsername}}}
  that your files were properly submitted.

  You can push more updates up until the deadline. Changes after the
  deadline will not be taken into account for grading.
\end{enumerate}
Homeworks must be legible and intelligible and on-time or may  be
returned ungraded with 0 points.

\paragraph{Bonus problems}

This assignment contains \textbf{bonus problems}. A bonus problem is
optional. If you do it you get additional points that count towards
this homework's total, although you can't get more than the maximum
number of points. If you don't do it you can still get full
points. Bonus problems and bonus points are indicated with an asterisk
``*''.

\paragraph{Included code and tests}

The homework comes with starter code in the \texttt{Submission}
directory. Edit and submit code as directed in the problems. The
directory also includes a file \texttt{test\_hw4.py}. You can use these
tests to check if your solutions are correct:
\begin{minted}{bash}
pytest -v test_hw4.py
\end{minted}
(If you solved all coding problems, you should see ``9 passed''; if
you also solved the Bonus problem \ref{sec:numpyfuncs}\ref{li:zeta}
you should see ``7 xpassed''. Otherwise you will be informed which
problems failed.)


\subsection{NumPy arrays (11 points)}

Work through the
\href{https://docs.scipy.org/doc/numpy-dev/user/quickstart.html}{NumPy
  tutorial}.\footnote{Some stuff such as the \texttt{ix\_()} function
  is fairly esoteric for beginners but almost everything else is what
  you should be familiar with for your daily work with arrays.} Do
the examples while you read it.

\begin{enuma}
\item How do NumPy array operations such as $+$, $-$, $*$, $/$ \dots
    differ from linear algebra operations (i.e. scalar product,
    vector/matrix multiplication, ...)? \points{2}
\item For the following, add your code to the file
  \texttt{problem1.py}. Given the three arrays
\begin{minted}{python}
import numpy as np

sx = np.array([[0, 1], [1, 0]])
sy = np.array([[0, -1j],[1j, 0]])
sz = np.array([[1, 0], [0, -1]])
\end{minted}
  \begin{enumi}
  \item What is the result of \texttt{result1b1 = sx * sy * sz}?
    Explain what NumPy array multiplication does to the arrays. (Note:
    your code should assigne the result to the variable
    \texttt{result1b1} in \texttt{problem1.py}.) \points{2}
  \item Use \texttt{np.dot()} to multiply the three arrays (like
    $\sigma_{x} \cdot \sigma_{y} \cdot \sigma_{z}$). Add your code to
    \texttt{problem1.py} and assign your result to variable
    \texttt{result1b2} Show your result and explain what
    happened. \points{2}
  \item Compute the ``commutator''
    $[\sigma_{x}, \sigma_{y}] := \sigma_{x}\cdot\sigma_{y} -
    \sigma_{y}\sigma_{x}$ and show that it equals
    $2i\sigma_{z}$.\footnote{These are the Pauli matrices that
      describe the three components of the spin operator for a spin
      $1/2$ particle,
      $\hat{\vec{S}} = \frac{\hbar}{2}\boldsymbol{\sigma}$. The fact
      that any two components of the spin operator do \emph{not}
      commute is a fundamental aspect of quantum mechanics.} Add your
    code to \texttt{problem1.py}, assign the result to variable
    \texttt{result1b3}. \points{3}
  \item Given a ``state vector''
    \begin{gather*}
      \vec{v} = \frac{1}{\sqrt{2}}\left(
        \begin{array}{c}
          1\\
          -i
        \end{array}\right)
    \end{gather*}
    calculate the ``expectation value''
    $\vec{v}^{\dagger} \cdot \sigma_{y} \cdot \vec{v}$ (i.e., the
    multiplication of the hermitian conjugate of the vector,
    $\vec{v}^{\dagger}$ with the matrix $\sigma_{y}$ and the vector
    $\vec{v}$ itself) using NumPy. \footnote{The \emph{hermitian
        conjugate} $\vec{v}^{\dagger} = (v_{1}^{*}, v_{2}^{*})$ is
      \texttt{v.conjugate().T} where \texttt{v.T} is shorthand for
      \texttt{v.transpose()}. It turns out that you don't need the
      transposition when you use \texttt{np.dot()} but I include it
      here for conceptual clarity. (Including \texttt{transpose()}
      comes at a minor performance penalty --- check with
      \texttt{\%timeit} if you are curious.)}  Add your code to
    \texttt{problem1.py} and assign your result to variable
    \texttt{result1b4}.\footnote{Note for anyone having done PHY 315
      (Quantum Mechanics II) that here you are calculating the quantum
      mechanical expectation value of the $y$-component of a spin
      $\frac{1}{2}$ particle in an eigenstate of the operator of the
      $y$-component of the spin ($\sigma_{y}\vec{v} = -\vec{v}$) and
      because $\vec{v}$ is normalized, you should get the eigenvalue
      as the expectation value.})  \points{2}
    \end{enumi}
\end{enuma}


\subsection{Coordinate manipulation with NumPy (16 points)}
We can represent the cartesian coordinates
$\vec{r}_{i} = (x_{i}, y_{i}, z_{i})$ for four particles as a numpy array
\texttt{positions}:
\begin{minted}{python}
  import numpy as np
  positions = np.array(\
      [[0.0, 0.0, 0.0], [1.34234, 1.34234, 0.0], \
       [1.34234, 0.0,  1.34234], [0.0, 1.34234, 1.34234]])
  t = np.array([1.34234, -1.34234, -1.34234])
\end{minted}
and \texttt{t} will be a translation vector.
\emph{For the following use NumPy.} Add your code to file
\texttt{problem2.py} and assign results to variables as indicated in
the problems.
\begin{enuma}
  \item What is the \emph{shape} of the array \texttt{positions} and
    what is its \emph{dimension}? \points{1}
  \item What is the \emph{shape} of the array \texttt{t} and
    what is its \emph{dimension}? \points{1}
  \item How do you access the coordinates of the second particle in
    \texttt{positions}? Assign the result to variable
    \texttt{result2c}. \points{1}
  \item For the second particle:
    \begin{enumi} 
    \item How do you access its $y$-coordinate? Assign the result to variable
    \texttt{result2d}.
      \points{2}
    \item What type of object is this output, what is its \emph{shape}
      and its \emph{dimension}? \points{2}
    \end{enumi}
  \item \label{li:nptranslation} Write Python code to translate all particles by a vector
    $\vec{t} = (1.34234, -1.34234, -1.34234)$,
\begin{minted}{python}
t = np.array([1.34234, -1.34234, -1.34234])
\end{minted}
    Add your code to \texttt{problem2.py} and assign the translated
    coordinates to variable \texttt{result2e}. \points{3}
  \item Make your solution of \ref{li:nptranslation} a function
    \texttt{translate(coordinates, t)}, which translates all
    coordinates in the argument \texttt{coordinates} (an
    \texttt{np.array} of shape \texttt{(N, 3)}) by the translation vector
    in \texttt{t}. The function should return the translated
    coordinates as a numpy array.

    Add the function to \texttt{problem2.py}. Show the results of the
    function applied to (1) the input \texttt{positions} and
    \texttt{t} from above and (2) for \texttt{positions2 =
      np.array([[1.5, -1.5, 3], [-1.5, -1.5, -3]])} and \texttt{t =
      np.array([-1.5, 1.5, 3])}. \points{6}
\end{enuma}


\subsection{NumPy functions (6 points)}
\label{sec:numpyfuncs}

\begin{enuma}
\item We want to plot the function \footnote{This is the definition
    used in
    \href{https://docs.scipy.org/doc/numpy/reference/generated/numpy.sinc.html}{numpy.sinc}
    function.}
  \begin{gather}
    \label{eq:sinc}
    \text{sinc}(x) := \frac{\sin(\pi x)}{\pi x}
  \end{gather}
  over the range $-6 \leq x < 6$.
  \begin{itemize}
  \item Use the NumPy \texttt{arange()} function to generate an array
    \texttt{X} with values from -6 to 6 in steps of 0.2.\footnote{You
      don't have to include the upper endpoint 6 in the range because
      this can be difficult to achieve with \texttt{arange()} and a
      floating point \texttt{step}; as an alternative you can look
      into using
      \href{https://docs.scipy.org/doc/numpy/reference/generated/numpy.linspace.html}{\texttt{numpy.linspace()}}.}.
  \item Apply the NumPy \texttt{sinc()} function to the \texttt{X}
    array\footnote{You do \emph{not} need any loops. Try
      \texttt{numpy.sinc(X)} and embrace NumPy!}
    and assign it to a variable \texttt{Y}.
  \item Plot your data with matplotlib
     \begin{minted}[autogobble]{python}
     import matplotlib.pyplot as plt
     plt.plot(X, Y)
     plt.xlabel("x")
     plt.ylabel("y = sinc(x)")
     plt.savefig("sinc.png")    # write plot to file
     \end{minted}
  \end{itemize}
  Submit your code as file \texttt{problem3a.py} together with the
  plot in file \texttt{sinc.png}. \points{4}
\item \label{li:zeta2}Use the NumPy \texttt{arange()}, the \texttt{sum()}, and the
  \texttt{sqrt()} functions to calculate the sum\footnote{You can
    compare your result to the analytical solution
    \begin{gather*}
      \sum_{k=1}^{\infty} \frac{1}{k^{2}} = \frac{\pi^{2}}{6}.
    \end{gather*}
  }
  \begin{gather}
    \label{eq:sum}
    S = \sqrt{\sum_{n=1}^{100} \frac{6}{n^{2}}}.
  \end{gather}
  Put your code into file \texttt{problem3b.py} and assign the result
  to a variable \texttt{mypi}. \points{2}
\item \label{li:zeta}\BONUS: Write a function to approximate the real-valued
  \href{http://mathworld.wolfram.com/RiemannZetaFunction.html}{Riemann
    Zeta function}
  \begin{gather}
    \label{eq:zeta}
    \zeta(s) = \sum_{k=1}^{\infty} \frac{1}{k^{s}}
  \end{gather}
  with the finite sum up to $N_{\text{max}}$ as
  \begin{gather}
    \label{eq:zetafinite}
    \zeta(s) \approx \zeta(s; N_{\text{max}}) := \sum_{k=1}^{N_{\text{max}}} \frac{1}{k^{s}}
  \end{gather}
  Add your function \texttt{zeta(s, Nmax=1000)} to a file
  \texttt{problem3c.py} and plot $\zeta(s)$ in the range
  $1 < s \leq 10$ and for a range of $N_{\text{max}}$ and include a
  figure of the plot \texttt{zeta.png}.\footnote{Note that the sum in
    problem \ref{li:zeta2}
    $\sum_{k=1}^{\infty} \frac{1}{k^{2}} = \frac{\pi^{2}}{6}$ is equal
    to $\zeta(2)$\dots} \bonus{4}
  
  You can plot multiple graphs within the same plot and add a legend
  with matplotlib:
  \begin{minted}[autogobble]{python}
     import numpy as np 
     import matplotlib.pyplot as plt
     
     def zeta(s, Nmax=1000):
         """Approximation to the real Riemann zeta function"""
         # add your code ...

     s = np.arange(1, 10, 0.1)
     Nmax_values = np.array([10, 100, 1000, 100000])

     fig = plt.figure(figsize=(6, 6))   # new figure          
     ax = fig.add_subplot(111)          # add "axes", i.e., graph to figure

     # plot each graph into the same axes and label by Nmax
     for Nmax in Nmax_values:
        ax.plot(s, zeta(s, Nmax=Nmax), label="Nmax="+str(Nmax))

     # finish the axes by adding labels and legend
     ax.set_xlabel(r"$s$")         # fancy LaTeX typeset label
     ax.set_ylabel(r"$\zeta(s)$")  # fancy LaTeX typeset label
     ax.legend(loc="best")         # place legends

     fig.savefig("zeta.png")
   \end{minted}
  (Add the \texttt{label="something"} keyword to the
  \href{https://matplotlib.org/api/_as_gen/matplotlib.axes.Axes.plot.html#matplotlib.axes.Axes.plot}{\texttt{plot()}}
  method, plot everything into the same graph (called an
  ``\href{https://matplotlib.org/api/axes_api.html#axes-class}{axes}''
  in matplotlib), and then call the
  \href{https://matplotlib.org/api/_as_gen/matplotlib.axes.Axes.legend.html#matplotlib.axes.Axes.legend}{\texttt{legend()}}
  method and save the figure.)
\end{enuma}



\subsection{BONUS: File processing in Python (15* bonus points)}
The standard way to open a file in Python and to process it line by
line is the code pattern
\begin{minted}[linenos,breaklines]{python}
with open(filename) as inputfile:
   for line in inputfile:
       line = line.strip()   # strip trailing/leading whitespace
       if not line:
          continue           # skip empty lines
       # now do something with a line
       # E.g., split into fields on whitespace
       fields = line.split()
       # access data as fields[0], fields[1], ... 
       x = float(fields[0])  # convert text to a float 
       y = float(fields[1])
       # ...
print("Processed file ", filename)
\end{minted}
In brief: 
\begin{enumerate}
\item A file is opened for reading with \texttt{open(filename)}, which
  returns a \emph{file object} (here assigned to the variable
  \texttt{inputfile}). The \texttt{with} statement is a very
  convenient way to make sure that the file is always being closed at
  the end: when the \texttt{with}-block exits (here at the
  \texttt{print} statement), \texttt{inputfile.close()} is called
  implicitly\cprotect\footnote{If you were not to use \texttt{with},
    your code would look like
\begin{minted}{python}
inputfile = open(filename)
for line in inputfile:
   # ...
inputfile.close()
print("Processed file ", filename)
\end{minted}
    but with the disadvantage that when something goes wrong during
    the \texttt{for}-loop, your file will never be closed, which
    exhausts system resources. When open a file for \emph{writing}
    (\texttt{open(filename, 'w')}) you will \emph{corrupt the file}
    when you are not closing it properly. The \texttt{with} statement
    guarantees that the file will \emph{always be closed, no matter
      what else happens}. Use the \texttt{with} statement!}.
\item We \emph{iterate} over all lines in the file (similar to what we
  did for lists) in a \texttt{for}-loop.
\item Remove leading and trailing white space with the
  \texttt{strip()} method of a string (\texttt{line} is a string). If
  you want to keep all white space, do not use \texttt{strip()}.
\item Skip empty lines: note that an empty string evaluates to
  \texttt{False} and thus can be used directly in the \texttt{if}
  statement. The \texttt{continue} statement then starts the next
  iteration in the loop.
\item Start processing the line. Often you know the structure of the
  file (e.g. a data file with 3 columns, separated by white space) so
  you typically split into fields (the string's \texttt{split()}
  method produces a list). Select the fields as needed.
\item As an example, fields 0 and 1 are assumed to represent floating
  point numbers. \texttt{fields[0]} contains a string but using
  \texttt{float(fields[0])} it can be converted (``cast'') to a Python
  float. Similarly, integer numbers can be cast with \texttt{int()}.
\end{enumerate}

Use the above information to write a Python program
\texttt{evaluate\_ships.py} that reads the file
\begin{center}
  \texttt{PHY494-resources/01\_shell/data/starships.csv}
\end{center}
splits lines on commas\footnote{``csv'' stands for ``comma separated
  values'' and is a common file format for tabular data.}, and prints
out the names and cost (in credits, ``CR'') of all starships that cost
more than 100 million CR.\footnote{Hint: Turn all \texttt{"unknown"}
  entries into 0 and then cast numbers to floats.}

Submit your code \texttt{evaluate\_ships.py} and your output in a file
\texttt{starship\_costs.dat}.  \bonus{15}




%Total Points: \arabic{TotalPoints}. Total Bonus: \arabic{TotalBonus}*


\end{document}

%%% Local Variables: 
%%% mode: latex
%%% TeX-master: t
%%% End: 
