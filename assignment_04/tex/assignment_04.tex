%%% generic article type (pdf)latex file
%%% use together with Makefile

\documentclass[letterpaper]{scrartcl}
\usepackage{graphicx}
\usepackage{amsmath,amsfonts,amsthm}
\usepackage{eufrak}
\usepackage{mathabx}
\usepackage{url}
\usepackage[usenames,dvipsnames,svgnames,table]{xcolor}
\usepackage[colorlinks]{hyperref}
\hypersetup{
     colorlinks   = true,
     urlcolor     = blue,
     linkcolor    = red,
     citecolor    = black
}
\usepackage{enumitem}
\usepackage{booktabs}
\usepackage{cprotect}
\usepackage{minted}


%\usepackage{wrapfig}
%\usepackage{subfig}
%\usepackage[format=plain,labelsep=period,font=small,labelfont=bf]{caption}

%------------------------------------------------------------
% assignment
%
\newcommand{\anumber}{4}
%
%------------------------------------------------------------

% hyperref https://en.wikibooks.org/wiki/LaTeX/Hyperlinks#.5Chref
\urlstyle{same}

%% not working yet...
\newcounter{TotalPoints}
\newcounter{TotalBonus}

\newcommand{\BONUS}{\textsc{Bonus: }}
\newcommand{\bonus}[1]{\textbf{[bonus +#1*]}\stepcounter{TotalBonus}}
\newcommand{\points}[1]{\textbf{[#1 points]}\stepcounter{TotalPoints}}
\newenvironment{enuma}{\begin{enumerate}[label=(\alph*)]}{\end{enumerate}}
\newenvironment{enumi}{\begin{enumerate}[label=(\roman*)]}{\end{enumerate}}
\newenvironment{solution}{\par\noindent\P{} }{\ \qedsymbol}

\renewcommand{\vec}[1]{\ensuremath{\mathbf{#1}}}
\newcommand{\pd}[3][]{\left(\frac{\partial #2}{\partial #3}\right)_{#1}}

\newcommand{\anum}{0\anumber}




\begin{document}
%\maketitle

\setcounter{section}{\anumber}
\addtocounter{section}{-1}
\section{ --- PHY 494: Homework assignment (36 points total)}

\noindent Due Thursday, Feb 13, 2020, 5pm.

\noindent
\fbox{\parbox{\linewidth}{Submission is now to your \textbf{private
      GitHub repository}. Follow the link provided to you by the
    instructor in order for the repository to be set up: It will have
    the name
    \emph{ASU-CompMethodsPhysics-PHY494/assignments-2020-\emph{YourGitHubUsername}}
    and will only be visible to you and the instructor/TA. Follow the
    instructions below to submit this (and all future) homework.}}
Read the following instructions carefully. Ask if anything is unclear.
\begin{enumerate}
\item \texttt{git clone} your assignment repository (change
  \emph{YourGitHubUsername} to your GitHub username)
  \begin{minted}{bash}
    repo="assignments-2020-YourGitHubUsername.git" 
    git clone https://github.com/ASU-CompMethodsPhysics-PHY494/${repo}
  \end{minted}
\item run the script
  \texttt{./scripts/update.sh} (replace \emph{YourGitHubUsername} with
  your GitHub username):
  \begin{minted}{bash}
    cd ${repo} 
    bash ./scripts/update.sh
  \end{minted}
  It should create three subdirectories\footnote{If the script fails,
    file an issue in the
    \href{https://github.com/ASU-CompMethodsPhysics-PHY494/PHY494-assignments-skeleton/issues}{Issue
      Tracker for PHY494-assignments-skeleton} and just create the
    directories manually.} \texttt{assignment\_\anum/Submission},
  \texttt{assignment\_\anum/Grade}, and
  \texttt{assignment\_\anum/Work}.
\item You can try out code in the \texttt{assignment\_\anum/Work}
  directory but you don't have to use it if you don't want to. Your
  grade with comments will appear in
  \texttt{assignment\_\anum/Grade}.
\item Create your solution in
  \texttt{assignment\_\anum/Submission}. Use Git to \texttt{git
    add} files and \texttt{git commit} changes.

  You can create a PDF, a text file or Jupyter notebook inside the
  \texttt{assignment\_\anum/Submission} directory as well as Python
  code (if required). \textbf{Name your files \texttt{hw\anum.pdf} or
    \texttt{hw\anum.txt} or \texttt{hw\anum.ipynb}}, depending on how
  you format your work. Files with code (if requested) should be named
  exactly as required in the assignment.
\item When you are ready to submit your solution, do a final
  \texttt{git status} to check that you haven't forgotten anything,
  commit any uncommited changes, and \texttt{git push} to your GitHub
  repository. Check on \emph{your} GitHub repository web
  page\footnote{\texttt{https://github.com/ASU-CompMethodsPhysics-PHY494/assignments-2020-\emph{YourGitHubUsername}}}
  that your files were properly submitted.

  You can push more updates up until the deadline. Changes after the
  deadline will not be taken into account for grading.
\end{enumerate}
Homeworks must be legible and intelligible and on-time or may  be
returned ungraded with 0 points.

\paragraph{Bonus problems}

This assignment contains \textbf{bonus problems}. A bonus problem is
optional. If you do it you get additional points that count towards
this homework's total, although you can't get more than the maximum
number of points. If you don't do it you can still get full
points. Bonus problems and bonus points are indicated with an asterisk
``*''.

\paragraph{Included code and tests}

The homework comes with starter code in the \texttt{Submission}
directory. Edit and submit code as directed in the problems. The
directory also includes files \texttt{test\_hw4.py} and
\texttt{test\_bugs.py}. You can use these tests to check if your
solutions are correct:
\begin{minted}{bash}
pytest
\end{minted}
(If you solved all coding problems, you should see ``112
passed''. Otherwise you will be informed which problems failed.)

\subsection{Counting Vowels (11 points)}
\label{sec:vowels}

Given a string \mintinline{python}{s}, count how often each of the 6 vowel letters
in the English alphabet (A, E, I, O, U, Y -- we include Y here)
occurs. You can ignore case by converting the string to lowercase with
\mintinline{python}{s.lower()}.

\begin{enuma}
\item Write a function \mintinline{python}{count_vowels(s)} and put it in a
  file \texttt{problem1.py}. It should take a string \mintinline{python}{s} as
  input and return a list (let's call it \mintinline{python}{counts}) with 6
  elements, where \mintinline{python}{count[0]} is the count for letter A,
  \mintinline{python}{count[1]} for E etc. \footnote{Hint: you can iterate through
    a string like a list using \mintinline{python}{for letter in s:} and
    then analyze the letter.}\points{10}
\item Apply your function to the string
  \begin{minted}{python}
    s = """'But I don't want to go among mad people,' Alice remarked.
    'Oh, you can't help that,' said the Cat, 'we're all mad here. I'm
    mad. You're mad.' 'How do you know I'm mad?' said Alice.  'You
    must be,' said the Cat, 'or you wouldn't have come here.'"""
  \end{minted}
  and report the counts in the variable \mintinline{python}{counts}
  in the same file. \points{1}
\end{enuma}


\subsection{Factorial Fun (15 points)}
\label{sec:factorialfun}

The factorial function is defined for integers $n>0$ by
\begin{gather}
  \label{eq:factorial}
  n! = n \cdot(n-1) \cdot (n-2) \dots 2 \cdot 1 = \prod_{k=1}^{n}k
\end{gather}
and $0! = 1$ is \emph{defined} for consistency.

The \href{http://mathworld.wolfram.com/DoubleFactorial.html}{double
  factorial} is defined by
\begin{gather}
  n!! = \begin{cases}
    n \cdot (n-2) \cdot (n-4) \cdot \dots \cdot 5 \cdot 3 \cdot 1,& \quad n>0 \ \text{odd}\\
    n \cdot (n-2) \cdot (n-4) \cdot \dots \cdot 6 \cdot 4 \cdot 2,& \quad n>0 \ \text{even}\\
    1,& \quad n = 0, -1
  \end{cases}.\label{eq:doublefactorial}  
\end{gather}

\begin{enuma}
\item Find a function in the
  \href{https://docs.python.org/3/library/math.html}{\mintinline{python}{math}}
  standard library to calculate $n!$ (Eq.~\ref{eq:factorial}).
  \begin{enumi}
  \item Write a function \mintinline{python}{factorial_math(n)} that
    computes $n!$ for $n \ge 0$. This function \emph{should} use the
    function from \mintinline{python}{math}. Put the function
    \mintinline{python}{factorial_math(n)} in a file
    \texttt{problem2a.py}. \points{2}
  \item Show results for the integers $n = 0, 1, 2, 3, ...,
    20$. \points{1}  
  \end{enumi}
\item Write a function to calculate the factorial $n!$ (Eq.~\ref{eq:factorial}):
  \begin{enumi}
  \item Create a function \mintinline{python}{factorial(n)} that
    computes $n!$ for $n \ge 0$. This function should \emph{not} use
    any modules (i.e., neither \mintinline{python}{math} nor
    \mintinline{python}{numpy}). Put the function in a file
    \texttt{problem2b.py}. \points{4}
  \item Show results for the integers $n = 0, 1, 2, 3, ...,
    20$. \points{1}
  \end{enumi}
\item Write a function to calculate the double factorial $n!!$
  (Eq.~\ref{eq:doublefactorial}):
  \begin{enumi}
  \item Create a function \mintinline{python}{double_factorial(n)}
    that computes the double factorial Eq.~\ref{eq:doublefactorial}
    for an integer \mintinline{python}{n}. The function should only return the
    value of $n!!$. This function should \emph{not} use any modules
    (i.e., neither \mintinline{python}{math} nor
    \mintinline{python}{numpy}). Put the function in a file
    \texttt{problem2c.py}. \points{6}
  \item Show results for the integers $n = 0, 1, 2, 3, ...,
    20$. \points{1}
  \end{enumi}
\end{enuma}


\subsection{Squash the bug (10 points)}
\label{sec:bugs}

Three files \texttt{bug\_a.py}, \texttt{bug\_b.py}, and
\texttt{bug\_c.py} are include. Each contains at least one Python
bug. Fix them and commit your fixed files.\footnote{You will also see a file
  \texttt{test\_bugs.py}. It contains \emph{tests} that check
  your code. Your instructors will \emph{run these tests on your
    code}. You can run them yourself with the \texttt{pytest} command,

  \mintinline{bash}{pytest -v test_bugs.py}

  If everything is correct, you should see something like
  \texttt{===== 36 passed in 0.36 seconds =====}. If tests fail then
  you can correct your code until you get the tests to pass.}
\begin{enuma}
\item Fix \texttt{bug\_a.py} and commit the fixed file. The code
  should run, assign the correct value to the variable \texttt{value},
  and print the correct value. \points{3}
\item Fix \texttt{bug\_b.py} and commit the fixed file. The code
  should add two vectors $\vec{a}= (12.3, 3.90, 4.5)$ and
  $\vec{b} = (1.3, 0.91, -3.3)$ and print the value of the new vector
  $\vec{c} = 5\vec{a} - 3\vec{b}$ and assign the value to the variable
  \texttt{c}. \points{3}
\item Fix \texttt{bug\_c.py} and commit the fixed file. You should
  correctly implement the 2D $\mathrm{sinc}$ function
  \begin{gather*}
    \mathrm{sinc}(x, y) := \frac{\sin x}{x} \frac{\sin y}{y}
  \end{gather*}
  (where $x$ and $y$ are angles given in radians). The function should be
  correct for arbitrary input.\footnote{Hint: Especially consider the edge
    and corner cases.} \points{4}
\end{enuma}

%%% Local Variables:
%%% mode: latex
%%% TeX-master: t
%%% End:


\subsection{BONUS: File processing in Python (15* bonus points)}
The standard way to open a file in Python and to process it line by
line is the code pattern
\begin{minted}[linenos,breaklines]{python}
with open(filename, 'r') as inputfile:
   for line in inputfile:
       line = line.strip()   # strip trailing/leading whitespace
       if not line:
          continue           # skip empty lines
       # now do something with a line
       # E.g., split into fields on whitespace
       fields = line.split()
       # access data as fields[0], fields[1], ... 
       x = float(fields[0])  # convert text to a float 
       y = float(fields[1])
       # ...
print("Processed file ", filename)
\end{minted}
In brief: 
\begin{enumerate}
\item A file is opened for reading with \mintinline{python}{open(filename, 'r')},
  which returns a \emph{file object} (here assigned to the variable
  \mintinline{python}{inputfile}).\footnote{Opening for writing uses the
    \mintinline{python}{'w'} argument and one can write a line to the file with
    \mintinline{python}{fileobject.write(...)}.} The
  \mintinline{python}{with} statement is a very convenient way to make
  sure that the file is always being closed at the end: when the
  \mintinline{python}{with}-block exits (here at the
  \mintinline{python}{print()} function),
  \mintinline{python}{inputfile.close()} is called
  implicitly\cprotect\footnote{If you were not to use \mintinline{python}{with},
    your code would look like
\begin{minted}{python}
inputfile = open(filename)
for line in inputfile:
   line = line.strip()   # strip trailing/leading whitespace
   # ... do more stuff
inputfile.close()
print("Processed file ", filename)
\end{minted}
    but with the disadvantage that when something goes wrong during
    the \mintinline{python}{for}-loop, your file will never be closed, which
    exhausts system resources. When open a file for \emph{writing}
    (\mintinline{python}{open(filename, 'w')}) you will \emph{corrupt the file}
    when you are not closing it properly. The \mintinline{python}{with} statement
    guarantees that the file will \emph{always be closed, no matter
      what else happens}. Use the \mintinline{python}{with} statement!}.
\item We \emph{iterate} over all lines in the file (similar to what we
  did for lists) in a \mintinline{python}{for}-loop.
\item Remove leading and trailing white space with the
  \mintinline{python}{strip()} method of a string (\mintinline{python}{line} is a string). If
  you want to keep all white space, do not use \mintinline{python}{strip()}.
\item Skip empty lines: note that an empty string evaluates to
  \mintinline{python}{False} and thus can be used directly in the \mintinline{python}{if}
  statement. The \mintinline{python}{continue} statement then starts the next
  iteration in the loop.
\item Start processing the line. Often you know the structure of the
  file (e.g. a data file with 3 columns, separated by white space) so
  you typically split into fields (the string's \mintinline{python}{split()}
  method produces a list). Select the fields as needed.
\item As an example, fields 0 and 1 are assumed to represent floating
  point numbers. \mintinline{python}{fields[0]} contains a string but using
  \mintinline{python}{float(fields[0])} it can be converted (``cast'') to a Python
  float. Similarly, integer numbers can be cast with \mintinline{python}{int()}.
\end{enumerate}

Use the above information to write a Python program
\texttt{evaluate\_ships.py} that reads the file
\href{https://github.com/ASU-CompMethodsPhysics-PHY494/PHY494-resources/blob/master/01_shell/data/starships.csv}{starships.csv}
(which is also provided as part of the homework). Fields in this file
are separated by \emph{commas}.\footnote{``csv'' stands for ``comma
  separated values'' and is a common file format for tabular data.}
The meaning of the fields (or columns) is
\begin{minted}{python}
name,model,vehicle_class,max_atmospheric_speed,cost_in_credits,length
\end{minted}
i.e, the first column contains the \emph{name}, the second column the
\emph{vehicle class} etc. Split the lines on commas and print out the
names and cost (in credits, ``CR'') of all starships that cost more
than 100 million CR.\footnote{Hint: Turn all \mintinline{python}{"unknown"}
  entries into 0 and then cast numbers to floats.}

Submit your code \texttt{evaluate\_ships.py} and your output in a file
\texttt{starship\_costs.dat}.  \bonus{15}




%Total Points: \arabic{TotalPoints}. Total Bonus: \arabic{TotalBonus}*


\end{document}

%%% Local Variables: 
%%% mode: latex
%%% TeX-master: t
%%% End: 
