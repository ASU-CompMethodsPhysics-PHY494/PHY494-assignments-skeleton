%%% generic article type (pdf)latex file
%%% use together with Makefile

\documentclass[letterpaper]{scrartcl}
\usepackage{graphicx}
\usepackage{amsmath,amsfonts,amsthm,amsbsy}
\usepackage{eufrak}
\usepackage{mathabx}
\usepackage{courier}
\usepackage{url}
\usepackage{color}
\usepackage[usenames,dvipsnames,svgnames,table]{xcolor}
\usepackage{hyperref}
\hypersetup{
     colorlinks   = true,
     urlcolor     = blue,
     linkcolor    = red,
     citecolor    = black
}


\usepackage{enumitem}
\usepackage{booktabs}
\usepackage{cprotect}
\usepackage{minted}

%\usepackage{wrapfig}
%\usepackage{subfig}
%\usepackage[format=plain,labelsep=period,font=small,labelfont=bf]{caption}

%------------------------------------------------------------
% assignment
%
\newcommand{\anumber}{10}
%
%------------------------------------------------------------
\newcommand{\anum}{\anumber}

% hyperref https://en.wikibooks.org/wiki/LaTeX/Hyperlinks#.5Chref
\urlstyle{same}


%% not working yet...
\newcounter{TotalPoints}
\newcounter{TotalBonus}

\newcommand{\BONUS}{\textsc{Bonus: }}
\newcommand{\bonus}[1]{\textbf{[bonus +#1*]}\stepcounter{TotalBonus}}
\newcommand{\points}[1]{\textbf{[#1 points]}\stepcounter{TotalPoints}}
\newenvironment{enuma}{\begin{enumerate}[label=(\alph*)]}{\end{enumerate}}
\newenvironment{enumi}{\begin{enumerate}[label=(\roman*)]}{\end{enumerate}}
\newenvironment{solution}{\par\noindent\P{} }{\ \qedsymbol}

\renewcommand{\vec}[1]{\ensuremath{\mathbf{#1}}}
\newcommand{\pd}[3][]{\left(\frac{\partial #2}{\partial #3}\right)_{#1}}




\begin{document}
%\maketitle

\setcounter{section}{\anumber}
\addtocounter{section}{-1}
\section{ --- PHY 494: Homework assignment (20 points total)}

\noindent Due Tuesday, April 7, 2020, \textbf{1:30pm}.

\noindent
%  \url{}
\fbox{\parbox{\linewidth}{Submission is to your \textbf{private
      GitHub repository}.}}
Read the following instructions carefully. Ask if anything is unclear.
\begin{enumerate}
\item \texttt{cd} into your assignment repository (change
  \emph{YourGitHubUsername} to your GitHub username) and run the
  update script \texttt{./scripts/update.sh} (replace
  \emph{YourGitHubUsername} with your GitHub username):
  \begin{minted}{bash}
    cd  assignments-2020-YourGitHubUsername
    bash ./scripts/update.sh
  \end{minted}
  It should create three subdirectories\footnote{If the script fails,
    file an issue in the
    \href{https://github.com/ASU-CompMethodsPhysics-PHY494/PHY494-assignments-skeleton/issues}{Issue
      Tracker for PHY494-assignments-skeleton} and just create the
    directories manually.} \texttt{assignment\_\anum/Submission},
  \texttt{assignment\_\anum/Grade}, and
  \texttt{assignment\_\anum/Work}.
\item You can try out code in the \texttt{assignment\_\anum/Work}
  directory but you don't have to use it if you don't want to. Your
  grade with comments will appear in
  \texttt{assignment\_\anum/Grade}.
\item Create your solution in
  \texttt{assignment\_\anum/Submission}. Use Git to \texttt{git add}
  files and \texttt{git commit} changes. Files with output and code
  should be named exactly as required in the assignment.
\item When you are ready to submit your solution, do a final
  \texttt{git status} to check that you haven't forgotten anything,
  commit any uncommited changes, and \texttt{git push} to your GitHub
  repository. Check on \emph{your} GitHub repository web
  page\footnote{\texttt{https://github.com/ASU-CompMethodsPhysics-PHY494/assignments-2020-\emph{YourGitHubUsername}}}
  that your files were properly submitted.

  You can push more updates up until the deadline. Changes after the
  deadline will not be taken into account for grading.
\end{enumerate}
Homeworks must be legible and intelligible or may otherwise be
returned ungraded with 0 points.


\paragraph{Proposal submission}

To submit your proposal (see below in Problem \ref{sec:proposal} for
detailed instructions on format and content), commit and \textbf{push
  a PDF file} inside the \texttt{assignment\_\anum{}/Submission}
directory and name it \texttt{proposal.pdf}.

\emph{Collaboration is not allowed} for this assignment. Each student
must produce their own proposal.


\paragraph{Project pitch}

On \textbf{Tuesday April 7, 2020}, you have your opportunity to
\emph{pitch your project to the class} and \emph{attract a team of
  students} to work with you on the project. \textbf{Only projects
  that were pitched to the class can be chosen!} You will have 5
Minutes in total to present you project in an exciting and succinct
manner.

If you want to \textbf{show slides} be ready to screen-share your
slides \footnote{Also include the presentation in your
  \texttt{assignment\_0\anumber{}/Submission} directory and submit it;
  it will not be graded but it is very convenient to have everything
  in one place and it will make it easy to get all presentations.}
However, slides are not required for the pitch --- you can use the
Zoom whiteboard or simply talk.


\subsection{Square-root with Newton's method (BONUS: 15* points)}
\label{sec:sqrt}

The square root function\footnote{The symbol $\sqrt{\cdot}$ is
  commonly used to denote the operation of $q$, so that
  $q(x) \equiv \sqrt{x}$. For the following it is more useful to think
  of the square root as a special function than just a calculation.}
$q(x)$ can be \emph{defined} by the equation
\begin{gather}
  \label{eq:sqrtdef}
  q(x)^{2} = x.
\end{gather}
The goal is to develop and to implement an \textbf{efficient algorithm
to compute square roots}.

The defining equation Eq.~\ref{eq:sqrtdef} can be rearranged as 
\begin{align}
  q(x)^{2} - x &= 0   \label{eq:sqrtroot}\\
  q^{2} - x &= 0 \label{eq:q}\\
  f_{x}(q) &= 0 \label{eq:fqx}
\end{align}
where $f_{x}(q) = q^{2} - x$ is now considered a function of the
\emph{variable} $q$ and a given \emph{parameter} $x$. Finding the
square root $q(x)$ amounts to finding the root of $f_{x}(q)$, i.e.,
find that value $q$ that makes Eq.~\ref{eq:q} true for a given $x$.

\begin{enuma}
\item \label{li:NR} Use the iterative Newton-Raphson algorithm from the class to
  implement a function \texttt{sqrt(x, tol=1e-6, Nmax=100)} in a
  module \texttt{functions.py} that returns the square root of $x$ to
  a tolerance of \texttt{tol} and uses at a maximum \texttt{Nmax}
  iterations. If \texttt{Nmax} is exceeded print a warning message and
  return \texttt{None}.\footnote{In your code you may \emph{not} use a
    library square root function such as \texttt{math.sqrt} or
    \texttt{numpy.sqrt} nor taking fractional powers such as
    \texttt{x**0.5} or \texttt{numpy.power(x, 0.5)}.} Your code should
  guess a good starting value for the Newton-Raphson scheme,
  e.g,. $x/2$. 

  In the Newton-Raphson scheme you have to calculate the update
  $\Delta q$ to $q$ in order to obtain the new best guess for the root
  \begin{gather}
    \label{eq:update}
    q \leftarrow q + \Delta q
  \end{gather}
  with
  \begin{gather}
    \label{eq:NR}
    \Delta q = -\frac{f_{x}(q)}{f_{x}'(q)}
  \end{gather}
  In your code you may either use a finite difference scheme to
  calculate $f_{x}'(q)$ \emph{or} (more efficiently), use the explicit
  \emph{analytical derivative} $f_{x}'(q) = \frac{df_{x}}{dq}$
  directly.\footnote{Using the analytical derivatives makes for a
    handy algorithm to \emph{manually} calculate square roots and
    indeed this is how Newton came up with the method. The algorithm
    for computing the square root was also already known to the
    ancient Babylonians.}

  Your code must produce correct results, as tested with
  \texttt{test\_functions.py}.\footnote{Some specific tests are
    allowed to fail and they are marked with \texttt{x} or
    \texttt{xfail} in the test output --- this is ok.} You can run
  these tests yourself with
\begin{minted}{bash}
pytest -v test_functions.py
\end{minted}
  (in the same directory as your \texttt{functions.py}). \bonus{10}
\item \label{li:results} Show results for
  $x = 0, 0.456\times 10^{-8}, 10^{-3}, 0.1, 0.64, 0.99, 1, 5, 9, 12.5,
  10^{3}, 1.2345 \times 10^{8}$
  for a tolerance of $10^{-6}$. Put the results in a text file
  \texttt{sqrt.txt}. The results should be arranged so that each line
  contains $x$ and $q(x)$. \bonus{5}
\item \label{li:analytical}\BONUS Do one Newton-Raphson step
  analytically (using $f'_{x}(q)$) and come up with a handy way to
  calculate approximations to square roots. Apply your approximation
  formula to estimate the value of $\sqrt{2}$ and compare to the value
  obtained from \texttt{numpy.sqrt()} or your calculator. \bonus{3}
\end{enuma}

\noindent
Note: You don't have to submit any notebooks or written
text for problems \ref{li:NR} and \ref{li:results}. It will be sufficient to submit
\begin{itemize}
\item \texttt{functions.py}
\item \texttt{sqrt.txt}
\end{itemize}
% If you are also solving the \emph{Bonus problem \ref{li:analytical}}
% then include a text document in PDF format to show your solution.
If you are also solving \emph{problem \ref{li:analytical}}
then include a text document in PDF format to show your solution.





\subsection{Proposal for the Final Project (20 points)}
\label{sec:proposal}

Write a short proposal for the \emph{Final Project}. You can come up
with your own idea or base your proposal on the list of suggestions in
the Appendix \ref{sec:suggestions}.\footnote{The Final Project, the
  proposal, and the suggested projects were discussed in class on
  2020-03-28. The slides are available from Canvas and in the PDF
  \texttt{final\_overview\_2020.pdf} within the HW\anumber{}
  repository.} Submit your proposal in \textbf{PDF format}.

\subsubsection{Formatting [5 points]}
\label{sec:formatting}

The proposal must adhere to the following formatting restrictions:
\begin{itemize}
\item 1 page maximum, 0.75 pages minimum length
\item minimum font size 11pt
\item minimum margins 1" on all sides
\item include
  \begin{itemize}
  \item \textbf{title}
  \item \textbf{author} (your name)
  \item sections with headings \textbf{Problem}, \textbf{Approach},
    and \textbf{Objectives} (in this order); see below for what the
    content should be.
  \end{itemize}
\end{itemize}

\subsubsection{Content [15 points]}
\label{sec:content}

The proposal should concisely describe what project you want to
undertake and suggest a roadmap for how you are going to do it. It
should be written in \emph{full English sentences}. Try to write clear
and succinctly. If you can read it to someone not in the class and they
get a general idea of what you want to do then you are on the right
track.

Specifically, you must address the following points in the individual
sections:
\begin{description}
\item[Problem] Describe the problem to be solved: What is the
  background, what is the overarching question. What is the physics
  governing the problem, what are the important equations---show them,
  if possible, but at a minimum name them. You can also comment
  on why this is an interesting or difficult problem. 

  Clearly define the overall goal of what you want to find out.

  \emph{This should be between 30-50\% of your text.}
\item[Approach] Describe \emph{how} you are going to reach your goal,
  i.e., answer the overarching question. How are you going to solve
  the equations that you identified in the Problem section, what kind
  of calculations are needed, which algorithms are you going to use?
  Do you need input parameters? Where can you get them from (show URLs
  or sources if possible).

  Be as concrete as possible: you want to convince your audience that
  it is feasible to solve this problem and you have an idea how to
  tackle it.

  \emph{This should be between 30-50\% of your text.}
  
\item[Objectives] Use a numbered list to state 3--6 measurable
  non-trivial outcomes that you need to achieve in order to reach the
  overall goal. These are the milestones that you have to reach; they
  are possibly dependent on each other. For each objective it must be
  clear how to decide if you fulfilled it or not. Objectives are often
  formulated in terms of deliverables such as ``Compile a list of
  material parameters for walls, windows and doors (heat conduction
  coefficients, typical thicknesses).''

  Your grade will partially depend on achieving the objectives that
  you set here.\footnote{The instructor will vet the objectives for any proposal
  that is being used as the basis for a Final Project.}

  \emph{This should take up your remaining space. The Objectives are
    very important. Make sure to formulate them clearly. They will be
    your own roadmap when you do the project.}  
\end{description}


\appendix

\section{Suggested projects}
\label{sec:suggestions}

The following are suggestions that can be used to develop
projects. The estimated difficulty level is only very approximate (1
being easiest and 4 hardest). However, it's even better if you come up
with your own projects. Don't be too ambitious: do something that
sounds like fun to you.

\subsection{Monte Carlo simulation of liquid argon}

Argon can be simulated as a \emph{Lennard-Jones} fluid. Using random
numbers and the \emph{Monte Carlo} (MC) approach we can simulate it at
constant temperature.

\begin{itemize}
\item difficulty: 1--2
\item implement basic MC for liquid Argon in the $NVT$ ensemble (see
  \emph{Computational Physics} and/or Frenkel and Smit,
  \emph{Understanding Molecular Simulations})
\item use periodic boundary conditions with the minimum image
  convention
\item analyze at different $T$ and calculate the equation of state
  $P(T, \rho)$; look for a phase transition
\item visualize
\item extra work: implement $NPT$ ensemble (with volume moves)
\end{itemize}


% \subsection{Monte Carlo simulation of the Ising magnet}

% The classical application for Monte Carlo methods and one of the best
% known hard but analytically solvable problems in statistical
% mechanics.

% \begin{itemize}
% \item difficulty: 2 (2D), 3 (with 3D)
% \item implement MC sampling for the Ising spin system in 1D, 2D, and
%   possibly 3D
% \item see \emph{Computational Physics} or \emph{Computational
%     Modelling and visualization of physical systems}
% \item compute magnetization $M(T)$ and look for a phase transition
% \item visualize
% \item compare to the exact 2D result (see eg Kerson Huang,
%   \emph{Statistical Mechanics})
% \end{itemize}

\subsection{Classical chaotic scattering}

Inspired by pinball flippers: what is needed for chaotic scattering,
i.e., small initial differences in an incoming beam lead to large
differences in the scattered beam.

\begin{itemize}
\item Difficulty: 1
\item Problem 9.4 in \emph{Computational Physics}
\item particle interacting with ``4-peak'' potential
\item integrate with RK4
\item vary parameters and analyze cross section, assess sensitivity
  to initial conditions
\item visualize trajectories
\end{itemize}



\subsection{Quantum mechanical wave packet propagation}

Simulate a realistic representation of particles interacting with
various potentials and see first-hand how different quantum mechanical
particles behave from classical ones.

\begin{itemize}
\item difficulty: 3
\item solve the time-dependent Schr\"odinger equation for a wave
  packet interacting with various potentials (e.g. using the explicit
  Visscher algorithm described in \emph{Computational Physics} 22.2.1
  or the Maestri/Askar \& Cakmak algorithm (\emph{Computational
    Physics} 22.3))
\item 1d
  \begin{itemize}
  \item step barrier (vary height compared to wave packet energy and
    observe tunneling)
  \item harmonic well
  \item square well
  \end{itemize}
\item 2d: observe interference \emph{for particles}:
  \begin{itemize}
  \item single slit
  \item double slit
  \end{itemize}
\item calculate transmission coefficient, wave
  velocity, probability density on a screen behind slits
\item visualize
\end{itemize}


\subsection{Fluid dynamics: 2D Navier-Stokes equations}

The Navier-Stokes equations govern all of atmospheric and ocean
physics, are needed to design airplanes and ships, and are fiendishly
non-linear. But you can solve them with a computer:

\begin{itemize}
\item difficulty: 2--3
\item Use \emph{Computational Physics} Chapter 25 with code as
  starting point
\item Solve the Navier-Stokes equations for 2d flow (using finite
  difference with successive over-relaxation)
\item investigate the velocity (flow) field and vorticity around various submerged
  objects: 
  \begin{itemize}
  \item beam
  \item sphere/cylinder
  \item drop shape
  \end{itemize}
\item visualize
\item vary Reynolds number: effect on vorticity?
\end{itemize}



\subsection{Dynamics of the solar system}

Simulate the whole solar system including a number of comets.

\begin{itemize}
\item difficulty: 2
\item simulate the solar system (classical gravitational interactions)
  with velocity Verlet
\item include sun and planets and comets such as Halley's comet or the
  Shoemaker-Levy comet and let it crash into Jupiter
\item find appropriate parameters (masses, positions,
  periods)\footnote{See NASA HORIZONS to get data:
    \url{http://ssd.jpl.nasa.gov/horizons.cgi}}
\item investigate
  \begin{itemize}
  \item stability of the solar system
  \item changes in orbits due to interactions (probably to weak to
    easily notice?)
  \item Do comets crash into planets?
  \item extra: course of small space craft (e.g. a gravity assist
    swing-by maneuver)
  \end{itemize}
\item extra: Make Jupiter like the sun and see how it changes the
  solar system.
\item extra: build the Tatooine system: binary star (two suns) and an
  earth-like planet (``Tatooine''). Does the planet have stable
  orbits? What would the days on the planet look like? Is there an
  orbit that allows life?
\item visualize with vpython\footnote{For a great example see the animated orrery
    \url{http://mgvez.github.io/jsorrery/}}
\end{itemize}



\subsection{Wavelet analysis of Stock markets or Oil prices}

Econophysics is a major employer of physicists\dots apply physical
ideas to markets. If you can predict where the prices go you can make a
lot of money\dots

See for instance the paper ``Wavelet-based prediction of oil
  prices'', S. Yousefi, I. Weinreich, D. Reinarz, \emph{Chaos,
    Solitons and Fractals} \textbf{25} (2005), 265--275, doi:
  \href{http://doi.org/10.1016/j.chaos.2004.11.015}{10.1016/j.chaos.2004.11.015}
\begin{itemize}
\item Difficulty: 4
\item Implement wavelet analysis (Daubechies) for stock prices and
  commodity prices time series. (see \emph{Computational Physics})
\item Test how  well you can forecast known data: correlate forecast
  vs actual data (which was not used for the wavelet analysis). How
  does the correlation coefficient of the forecast with the real
  timeseries vary as a function of the forecasting time interval?
\end{itemize}
Disclaimer: If you make money from this project, it's all yours.


% \subsection{Agent-based modelling of self-driving cars}
% \label{sec:cars}

% The future belongs to self-driving cars, which will communicate with
% each other. Once all cars are self-driving they might behave like a
% swarm of birds or bees or a school of fish: simple and local rules
% (such as keep a minimum distance, move if your neighbor moves, \dots)
% can lead to seemingly complex behavior in aggregate. Are self-driving
% cars going to improve traffic (e.g., increase traffic flow
% throughput)? How likely are accidents if small errors occur, i.e., how
% robust is the system?

% I heard something like the following in a TED talk:
% \begin{quotation}
%   ``Once cars can talk to each other, we will not need any traffic
%   lights or even restrictions on which side of the road they can
%   drive, it will look very organic, e.g. like water flowing.''
% \end{quotation}
% Is this true?

% \begin{itemize}
% \item difficulty: 2 (?)
% \item Use \emph{agent-based modelling} to simulate large numbers of
%   cars on a small street network. (Basically, many little car objects
%   that all follow their own rules in response to everything else in
%   the environment.)
% \item Analyse traffic flow as function of car density (and rules).
% \item Extension: introduce random errors or human drivers and study
%   robustness.
% \end{itemize}

\subsection{Analysis of natural motion}
\label{sec:motion}

When you look at a film of a human walking, a cheeta or a horse
racing, a bird flying, a fish swimming, or a snake slithering you get
24 pictures per second that are all different and seem to show a very
complicated process. However, using machine learning techniques such
as \emph{singular value decomposition} (SVD --- see lecture 14!) these
movements can often be decomposed into a small number of components and
one gets a quantitative handle on the motion. (Example: Girdhar K,
Gruebele M, Chemla YR (2015) The Behavioral Space of Zebrafish
Locomotion and Its Neural Network Analog. \emph{PLoS ONE} 10(7):
e0128668. doi:10.1371/journal.pone.0128668 )

\begin{itemize}
\item difficulty: 3--4 (?)
\item record or obtain video of animal/human motion and process it
  with Python
\item use SVD to decompose motions and assess the complexity
\item What are the most complex motions? Is human walking more complex
  than a fish swimming or an eagle flying? Do all birds fly the same
  way?
\item Extra: Perform ``gait analysis/recognition'' (also see
  \emph{Misson Impossible 5: Rogue Nation}\footnote{See
    \url{https://www.youtube.com/watch?v=0iZ-nQ4yFn4} starting at
    0:51.}): can the way you walk act as a unique finger print?
\end{itemize}


\end{document}

%%% Local Variables: 
%%% mode: latex
%%% TeX-master: t
%%% End: 
